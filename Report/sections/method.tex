\chapter{Method}
\label{sec:method}

\subsection{Physics-Informed Neural Networks}

Physics-informed neural networks (PINN) were introduced by M. Raissi et. al (citation here). 
They are neural networks designed to solve partial differential equations. 
Essentially you constrain a neural network using physical laws.

Create a DNN with param $\theta$
Have two training sets, one for the equ and one for BC/IC
Have a loss function that sums the loss from the pde and the bc
Find best $\theta$ that minimizes the loss function

Define a PDE problem, $u_{t}+\mathcal{N}[u]=0, x \in \Omega, t \in[0, T]$
Define $f:=u_{t}+\mathcal{N}[u]$
Approximate $u(t,x)$ with a DNN
$f(x,t)$ is thus a PINN 
Train by minimizing $M S E=M S E_{u}+M S E_{f}$, where $M S E_{u}=\frac{1}{N_{u}} \sum_{i=1}^{N_{u}}\left|u\left(t_{u}^{i}, x_{u}^{i}\right)-u^{i}\right|^{2}$ and $M S E_{f}=\frac{1}{N_{f}} \sum_{i=1}^{N_{f}}\left|f\left(t_{f}^{i}, x_{f}^{i}\right)\right|^{2}$

To implement PINNs we used the package DeepXDE by Lu Lu et. al (citation here). 


\subsection{Systems Biology Informed Deep Learning}

Further complication of a PINN. 
Adds several layers to the NN: Input-scaling layer, Feature layer, and Output-scaling layer.
Used to infer the hidden dynamics of experimentally unobserved species as well as the unknown parameters in the system of equations


\subsubsection{Importance of Correct Feature Transform}





\subsection{FitzHugh–Nagumo}

The FitzHugh–Nagumo model is considered one of the simplest cardiac cell models. It has two states and four variables.
%talk about each of the states and variables


\begin{align}\label{eq:fhn} %give only one label
    &\dot{v}=v-\frac{v^{3}}{3}-w+R I_{\mathrm{ext}} \\
    &\tau \dot{w}=v+a-b w
\end{align}


\subsubsection{Analysis of Stiffness for FHN}
%or instability

The stiffness index is often defined as (from Suyong Kim et al eq. 5):
\begin{align}
S = \frac{\operatorname{Re}\left(\lambda_{\max }\right)}{\operatorname{Re}\left(\lambda_{\min }\right)}\left(t_{1}-t_{0}\right),
\end{align}
where $\lambda_i$ are the eigenvalues of the Jacobian matrix of the ODE. From equation \ref{eq:fhn} we get the Jacobian
\begin{align}
J_f = \begin{pmatrix} %should it be pmatrix or bmatrix?
	1 - v^2 && -1 \\
	\frac{1}{\tau} && - \frac{b}{\tau} 
\end{pmatrix}.\end{align}
We then find the eigenvalues:
\begin{align}
0 &= \det (J_f - \lambda I) 
= \det \begin{vmatrix}
	1 - v^2 -\lambda && -1 \\
	\frac{1}{\tau} && - \frac{b}{\tau} -\lambda
\end{vmatrix} \\
&= \left( 1-v^2-\lambda\right)  \left( -\frac{b}{\tau} -\lambda \right) - \frac{1}{\tau} \\
&= -\frac{b}{\tau} - \lambda + v^2 \frac{b}{\tau} + v^2\lambda + \lambda \frac{b}{\tau} + \lambda^2 - \frac{1}{\tau} \\
&= \lambda^2 + \lambda \left( -1 + v^2 + \frac{b}{\tau} \right) + v^2\frac{b}{\tau} - \frac{b}{\tau} - \frac{1}{\tau} \\
&= \lambda^2 + \lambda \left( -1 + v^2 + \frac{b}{\tau} \right) + \frac{1}{\tau} \left(  v^2 b - b - 1 \right).
\end{align}

We can then use the quadratic formula
\begin{align}
\lambda &= \frac{- v^2 - \frac{b}{\tau} + 1}{2} \pm  \sqrt{\frac{ \left( v^2 + \frac{b}{\tau} -1 \right)^2 - \frac{4}{\tau} \left(  v^2 b - b - 1 \right) }{4}} \\
&= \frac{- v^2 - \frac{b}{\tau} + 1}{2} \pm  \sqrt{\frac{ v^4 + 2v^2 \frac{b}{\tau} - 2v^2 + \frac{b^2}{\tau^2} - 2 \frac{b}{\tau} + 1 - 4v^2 \frac{b}{\tau} + 4\frac{b}{\tau} + \frac{4}{\tau} }{4}} \\
&= \frac{- v^2 - \frac{b}{\tau} + 1  \pm  \sqrt{ \left(\frac{b}{\tau} - v^2 + 1 \right)^2 + \frac{4}{\tau} }}{2}
\end{align}
which gives us two values for $\lambda$. 




























