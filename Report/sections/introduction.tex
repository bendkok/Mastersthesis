\chapter{Introduction}
\label{sec:intro}

Metamorphic rocks compose \SI{27,4}{\%} of the Earth's crust by volume.
We can cite this claim using a totally unrelated source \parencite{Cor+14}.
Gold has a density of \SI{19,32}{\gram\per\centi\meter\cubed}.
We can back up this claim with two unrelated sources at the same time \parencite{DeB11, NGU19}.

We have a cross-correlation function given by
\begin{equation}
    \label{eq:cross-correlation}
    C(\omega)
    =
    \int_{0}^{2\pi} \exp \p*{i \frac{\omega r}{c} \cos \theta} \diff \theta.
\end{equation}
Applying the inverse Fourier transform to \eqref{eq:cross-correlation} yields
\begin{equation*}
    C(t)
    =
    \frac{1}{2\pi} \int_{-\infty}^{\infty} \int_{0}^{2\pi}
    \exp \p*{-i \frac{\omega r}{c} \cos \theta}
    \exp \p{i \omega t} \diff \theta \diff \omega.
\end{equation*}

\kant[4-6] % Dummy text

\section{Outline}

The rest of the text is organised as follows:
\begin{description}
    \item[\cref{sec:method}]
    is second to none, with the notable exception of \cref{sec:intro}.
    The main tool introduced here is the employment of unintelligible sentences.

    \item[\cref{sec:results}]
    asserts the basic properties of being the third chapter of a text.
    This section reveals the shocking truth of filler content.

    \item[\cref{sec:discussion}]
    demonstrates how easily one can get to four chapters by simply using
    the \texttt{kantlipsum} package to generate dummy words.
    
    \item[\cref{sec:conslusion}]
    demonstrates how easily one can get to four chapters by simply using
    the \texttt{kantlipsum} package to generate dummy words.

    \item[\cref{sec:first-app}]
    features additional material for the specially interested.

    \item[\cref{sec:second-app}]
    consists of results best relegated to the back of the document,
    ensuring that nobody will ever read it.
\end{description}