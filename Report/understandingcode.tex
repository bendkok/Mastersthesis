\documentclass[a4paper]{article}
% Import some useful packages
\usepackage[margin=0.6in]{geometry} % narrow margins
\usepackage[utf8]{inputenc}
\usepackage[english]{babel}
\usepackage{hyperref}
\usepackage{listings}
\usepackage{amsmath,graphicx,varioref,verbatim,amsfonts,geometry,amssymb,dsfont,blindtext}
%\usepackage{minted}
\usepackage{amsmath}
\usepackage{xcolor}
\usepackage{booktabs}
\usepackage{epstopdf}
\usepackage{media9}
\usepackage{float}
\hypersetup{colorlinks=true}
\definecolor{LightGray}{gray}{0.95}

\definecolor{dkgreen}{rgb}{0,0.55,0}
\definecolor{blue}{rgb}{0,0,0.8}
\definecolor{gray}{rgb}{0.5,0.5,0.5}
\definecolor{mauve}{rgb}{0.58,0,0.82}
\definecolor{red}{rgb}{0.8,0,0}
\definecolor{mygray}{rgb}{0.96,0.96,0.96}
\definecolor{LightGray}{gray}{0.95}
\newcommand{\code}[1]{\colorbox{lightgray}{\texttt{#1}}}

\lstset{frame=tb,
	language=Python,
	aboveskip=3mm,
	belowskip=3mm,
	showstringspaces=false,
	columns=flexible,
	basicstyle={\small\ttfamily},
	numbers=none,
	otherkeywords={self,np,plt},
	numberstyle=\tiny\color{mauve},
	identifierstyle=\color{black},
	keywordstyle=\color{blue},
	commentstyle=\color{dkgreen},
	stringstyle=\color{red},
	backgroundcolor=\color{mygray},
	rulecolor=\color{black},
	breaklines=true,
	breakatwhitespace=true,
	%tabsize=3
	extendedchars=true,
	literate=
	{á}{{\'a}}1 {é}{{\'e}}1 {í}{{\'i}}1 {ó}{{\'o}}1 {ú}{{\'u}}1
	{Á}{{\'A}}1 {É}{{\'E}}1 {Í}{{\'I}}1 {Ó}{{\'O}}1 {Ú}{{\'U}}1
	{à}{{\`a}}1 {è}{{\`e}}1 {ì}{{\`i}}1 {ò}{{\`o}}1 {ù}{{\`u}}1
	{À}{{\`A}}1 {È}{{\'E}}1 {Ì}{{\`I}}1 {Ò}{{\`O}}1 {Ù}{{\`U}}1
	{ä}{{\"a}}1 {ë}{{\"e}}1 {ï}{{\"i}}1 {ö}{{\"o}}1 {ü}{{\"u}}1
	{Ä}{{\"A}}1 {Ë}{{\"E}}1 {Ï}{{\"I}}1 {Ö}{{\"O}}1 {Ü}{{\"U}}1
	{â}{{\^a}}1 {ê}{{\^e}}1 {î}{{\^i}}1 {ô}{{\^o}}1 {û}{{\^u}}1
	{Â}{{\^A}}1 {Ê}{{\^E}}1 {Î}{{\^I}}1 {Ô}{{\^O}}1 {Û}{{\^U}}1
	{œ}{{\oe}}1 {Œ}{{\OE}}1 {æ}{{\ae}}1 {Æ}{{\AE}}1 {ß}{{\ss}}1
	{ç}{{\c c}}1 {Ç}{{\c C}}1 {ø}{{\o}}1 {å}{{\r a}}1 {Å}{{\r A}}1
	{€}{{\EUR}}1 {£}{{\pounds}}1
}


%\title{Understanding the Code}
%\author{Bendik Steinsvåg Dalen}
\renewcommand\thesubsection{\thesection.\alph{subsection}}
\renewcommand\thesubsubsection{\thesubsection.\roman{subsubsection}}
\begin{document}

[Work in progress].

\section{Intro}

We can think of the problem as trying to solve the optimization problem $min_x f(x)$ numerically. Here $f(x)$ would be the loss function,
\begin{align}
	f(x) = \mathcal{L}(\boldsymbol{\theta}, \mathbf{p})=\mathcal{L}^{\text {data }}(\boldsymbol{\theta})+\mathcal{L}^{\text {ode }}(\boldsymbol{\theta}, \mathbf{p})+\mathcal{L}^{\operatorname{aux}}(\boldsymbol{\theta}), 
\end{align}
where
\begin{align}
\mathcal{L}^{ \text{data}}(\boldsymbol{\theta}) &= \sum_{m=1}^{M} w_{m}^{\text {data }} \mathcal{L}_{m}^{\text {data }}=\sum_{m=1}^{M} w_{m}^{\text {data }}\left[\frac{1}{N^{\text {data }}} \sum_{n=1}^{N^{\text {data }}}\left(y_{m}\left(t_{n}\right)-\hat{x}_{s_{m}}\left(t_{n} ; \boldsymbol{\theta}\right)\right)^{2}\right] \\
\mathcal{L}^{\text {ode }}(\boldsymbol{\theta}, \mathbf{p}) &= \sum_{s=1}^{s} w_{s}^{\text {ode }} \mathcal{L}_{s}^{\text {ode }}=\sum_{s=1}^{s} w_{s}^{\text {ode }}\left[\frac{1}{N^{\text {ode }}} \sum_{n=1}^{N^{\text {ode }}}\left(\left.\frac{d \hat{x}_{s}}{d t}\right|_{\tau_{n}}-f_{s}\left(\hat{x}_{s}\left(\tau_{n} ; \boldsymbol{\theta}\right), \tau_{n} ; \mathbf{p}\right)\right)^{2}\right] \\
\mathcal{L}^{a u x}(\boldsymbol{\theta}) &= \sum_{s=1}^{s} w_{s}^{a u x} \mathcal{L}_{s}^{a u x}=\sum_{s=1}^{s} w_{s}^{\text {aux }} \frac{\left(x_{s}\left(T_{0}\right)-\hat{x}_{s}\left(T_{0} ; \boldsymbol{\theta}\right)\right)^{2}+\left(x_{s}\left(T_{1}\right)-\hat{x}_{s}\left(T_{1} ; \boldsymbol{\theta}\right)\right)^{2}}{2}.
\end{align}
(Let's get back to what these mean later).

\section{What x is}

The $x$ in our problem can be represented as $[\boldsymbol{\theta}, \mathbf{p}]$, where $\boldsymbol{\theta}$ is the internal parameters of the NN, while $\mathbf{p} = [a, b, \tau, I_{ext}]$ is the constants in our ODE (the FitzHugh–Nagumo model). 

\subsection{\textbf{p}}

In our code the ODE-constants are represented as:
\begin{lstlisting}[language=python]
a = tf.math.softplus(tf.Variable(0, trainable=True, dtype=tf.float32)) * .1
b = tf.math.softplus(tf.Variable(0, trainable=True, dtype=tf.float32))
tau = tf.math.softplus(tf.Variable(0, trainable=True, dtype=tf.float32)) * 10
Iext = tf.math.softplus(tf.Variable(0, trainable=True, dtype=tf.float32)) * .1
var_list = [a, b, tau, Iext]
\end{lstlisting}
The constants are represented using \lstinline|tf.Variable()|-function. This is a common way to represent parameters in TensorFlow, especially ones that can be differentiated or tuned.
I'm not sure what it does exactly, but I think it's a type of tensor with some extra properties. 

The \lstinline|softpluss()| function is a smooth approximation to ReLU, and scales the variable. I'm not sure why this function is used specifically though. 
The multiplications at the end are used to further scale the variables to be of the same order of magnitude as the values we are trying to approximate. 


\subsection{$\theta$}

Our neural network is represented as:
\begin{lstlisting}[language=python]
net = dde.maps.FNN([1] + [128] * 3 + [2], "swish", "Glorot normal")
\end{lstlisting}
All of $\boldsymbol{\theta}$ is contained in this variable. 

We thus have a NN with 1 input node, 3 hidden layers with 128 nodes, and two output nodes. 
I belive only one value of $t$ is used at a time, though I'm not entirely sure. 
%I assume all of $t$ is inserted into the input, otherwise I'm not sure why there's only one input node.

The NN uses the swish activation function,
\begin{align}
	\operatorname{swish}(x):=x \times \operatorname{sigmoid}(\beta x)=\frac{x}{1+e^{-\beta x}}.
\end{align}

The parameters are initialized using the Glorot normal initializer, which has a mean of 0 and a standard deviation of $\sqrt{\frac{2}{N_{\text{input nodes}} + N_{\text{output nodes}}}}$. It might be wort looking into other initialization methods. 

It also receives a input feature transform:
\begin{lstlisting}[language=python]
def feature_transform(t):
	return tf.concat(
		(
			t,
			tf.sin(0.01 * t),
			tf.sin(0.05 * t),
			tf.sin(0.1 * t),
			tf.sin(0.15 * t),
			#try f.exs. tf.sin(0.15 * t + 5),
		),
		axis=1,
	)
net.apply_feature_transform(feature_transform)
\end{lstlisting}
$t$ will thus go from having shape $[N, 1]$ to $[N, 7]$. 
This specific transformation is used to mimic the periodicity pattern in the yeast glycolysis model. Our model also has periodicity, so using $\sin(kt)$ is also likely to be a good idea, however different choices for $k$ (and a different number of $k$s) might be more appropriate. Leaving out the feature transforms might work as well. 

Finally the NN receives a output transformation:
\begin{lstlisting}[language=python]
def output_transform(t, y):
	return (
		data_y[0] + tf.math.tanh(t) * tf.constant([1., 1.]) * y
	)
net.apply_output_transform(output_transform)
\end{lstlisting}
If I've understood it correctly, this part essentially just scales the output.

The paper also mentions input scaling, however I haven't been able to find a specific place in the code where this is done.




\section{Loss function}

$f(x)$, or our loss function, has three parts: data, ODE and auxiliary loss. 

\subsection{Auxiliary}

The function for auxiliary loss is

\begin{align}
\mathcal{L}^{a u x}(\boldsymbol{\theta}) &= \sum_{s=1}^{S} w_{s}^{a u x} \mathcal{L}_{s}^{a u x}=\sum_{s=1}^{S} w_{s}^{\text {aux }} \frac{\left(x_{s}\left(T_{0}\right)-\hat{x}_{s}\left(T_{0} ; \boldsymbol{\theta}\right)\right)^{2}+\left(x_{s}\left(T_{1}\right)-\hat{x}_{s}\left(T_{1} ; \boldsymbol{\theta}\right)\right)^{2}}{2}.
\end{align}
The article describes $S$ as the amount of species in the system. I think that's the same as states in the ODE-system, which would be 2 for FHN. 
$x_s$ represents the value state $S$, $\hat{x}_{s}$ is the output of the NN, $T_0$ is the start time point, $T_1$ is an arbitrary time point, $\boldsymbol{\theta}$ are the internal variables of the NN and $w_{m}^{\text {data }}$ are the auxiliary weights.

I'm not sure exactly which part of the code is meant to be the auxiliary loss. It's either
\begin{lstlisting}[language=python]
# Create a time domain from first to last timepoint
geom = dde.geometry.TimeDomain(data_t[0, 0], data_t[-1, 0])
\end{lstlisting}
or
\begin{lstlisting}[language=python]
def boundary(x, _):
	return np.isclose(x[0], data_t[-1, 0])

y1 = data_y[-1]
bc0 = dde.DirichletBC(geom, lambda X: y1[0], boundary, component=0)
bc1 = dde.DirichletBC(geom, lambda X: y1[1], boundary, component=1)
\end{lstlisting}
or both. 



\subsection{Data}

The function for data loss is
\begin{align}
\mathcal{L}^{ \text{data}}(\boldsymbol{\theta}) &= \sum_{m=1}^{M} w_{m}^{\text {data }} \mathcal{L}_{m}^{\text {data }}=\sum_{m=1}^{M} w_{m}^{\text {data }}\left[\frac{1}{N^{\text {data }}} \sum_{n=1}^{N^{\text {data }}}\left(y_{m}\left(t_{n}\right)-\hat{x}_{s_{m}}\left(t_{n} ; \boldsymbol{\theta}\right)\right)^{2}\right],
\end{align}
where $N^{\text {data }}$ is the number of data points, $M$ is the number of measurements, $y_{1}, y_{2}, \ldots, y_{M}$ is our measurements at time $t_{1}, t_{2}, \ldots, t_{N^{d a t a}}$. $\hat{x}_{S_m}$ are the outputs of the NN, $w_{m}^{\text {data }}$ are the data weights, $\boldsymbol{\theta}$ are the internal variables of the NN, and $w_{m}^{\text {data }}$ are the data weights. 

For the FHN model we get $M=S$ since both the states are measurable. 






\subsection{ODE}

We enforce the network to satisfy the ODE system at $N^{\text {ode }}$ time points $\tau_{n}$.
The function for ODE loss then becomes 
\begin{align}
\mathcal{L}^{\text {ode }}(\boldsymbol{\theta}, \mathbf{p}) 
&= \sum_{s=1}^{S} w_{s}^{\text {ode }} \mathcal{L}_{s}^{\text {ode }}
= \sum_{s=1}^{S} w_{s}^{\text {ode }} \left[\frac{1}{N^{\text {ode }}} \sum_{n=1}^{N^{\text {ode }}}\left(\left. \frac{d \hat{x}_{s}}{d t}\right|_{\tau_{n}} - f_{s}\left(\hat{x}_{s}\left(\tau_{n} ; \boldsymbol{\theta}\right), \tau_{n} ; \mathbf{p}\right)\right)^{2}\right],
\end{align}
where $\hat{x}_{s}$ is the output of the NN for state $s$, $w_{m}^{\text {ode }}$ are the ODE weights and $f_s$ is the solution to the ODE system with the given parameters $\mathbf{p}$. For FHN $\mathbf{p} = [a, b, \tau, I_{ext}]$.

Automatic differentiation is used to compute $\left. \frac{d \hat{x}_{s}}{d t}\right|_{\tau_{n}}$. 


\subsection{Compiling the Loss Functions}





\subsection{Weights}

The weights are represented as such in the code:
\begin{lstlisting}[language=python]
bc_weights = [1, 1]
if noise >= 0.1:
	bc_weights = [w * 10 for w in bc_weights]
	
data_weights = [1, 1]
# Large noise requires small data_weights
if noise >= 0.1:
	data_weights = [w / 10 for w in data_weights]
	
ode_weights = [1, 1]
# Large noise requires large ode_weights
if noise > 0:
	ode_weights = [10 * w for w in ode_weights]
\end{lstlisting}

When the model is compiled the loss is added as such:
\begin{lstlisting}[language=python]
model.compile(
	"adam",
	lr=1e-3,
	loss_weights=[0] * 2 + weights["bc_weights"] + weights["data_weights"],
)
\end{lstlisting}
for first 1000 epochs. Here there's the ODE loss has no weight, and is thus ignored. 
For the remaining epochs the loss is added as such:
\begin{lstlisting}[language=python]
model.compile(
	"adam",
	lr=1e-3,
	loss_weights=weights["ode_weights"]
	+ weights["bc_weights"]
	+ weights["data_weights"],
)
\end{lstlisting}
















	
\end{document}

