\documentclass[a4paper]{article}
% Import some useful packages
\usepackage[margin=0.6in]{geometry} % narrow margins
\usepackage[utf8]{inputenc}
\usepackage[english]{babel}
\usepackage{hyperref}
\usepackage{listings}
\usepackage{amsmath,graphicx,varioref,verbatim,amsfonts,geometry,amssymb,dsfont,blindtext}
%\usepackage{minted}
\usepackage{amsmath}
\usepackage{xcolor}
\usepackage{booktabs}
\usepackage{epstopdf}
\usepackage{media9}
\usepackage{float}
\usepackage{caption}
\usepackage{subcaption}
\hypersetup{colorlinks=true}
\definecolor{LightGray}{gray}{0.95}

\definecolor{dkgreen}{rgb}{0,0.55,0}
\definecolor{blue}{rgb}{0,0,0.8}
\definecolor{gray}{rgb}{0.5,0.5,0.5}
\definecolor{mauve}{rgb}{0.58,0,0.82}
\definecolor{red}{rgb}{0.8,0,0}
\definecolor{mygray}{rgb}{0.96,0.96,0.96}
\definecolor{LightGray}{gray}{0.95}
\newcommand{\code}[1]{\colorbox{lightgray}{\texttt{#1}}}

\lstset{frame=tb,
	language=Python,
	aboveskip=3mm,
	belowskip=3mm,
	showstringspaces=false,
	columns=flexible,
	basicstyle={\small\ttfamily},
	numbers=none,
	otherkeywords={self,np,plt},
	numberstyle=\tiny\color{mauve},
	identifierstyle=\color{black},
	keywordstyle=\color{blue},
	commentstyle=\color{dkgreen},
	stringstyle=\color{red},
	backgroundcolor=\color{mygray},
	rulecolor=\color{black},
	breaklines=true,
	breakatwhitespace=true,
	%tabsize=3
	extendedchars=true,
	literate=
	{á}{{\'a}}1 {é}{{\'e}}1 {í}{{\'i}}1 {ó}{{\'o}}1 {ú}{{\'u}}1
	{Á}{{\'A}}1 {É}{{\'E}}1 {Í}{{\'I}}1 {Ó}{{\'O}}1 {Ú}{{\'U}}1
	{à}{{\`a}}1 {è}{{\`e}}1 {ì}{{\`i}}1 {ò}{{\`o}}1 {ù}{{\`u}}1
	{À}{{\`A}}1 {È}{{\'E}}1 {Ì}{{\`I}}1 {Ò}{{\`O}}1 {Ù}{{\`U}}1
	{ä}{{\"a}}1 {ë}{{\"e}}1 {ï}{{\"i}}1 {ö}{{\"o}}1 {ü}{{\"u}}1
	{Ä}{{\"A}}1 {Ë}{{\"E}}1 {Ï}{{\"I}}1 {Ö}{{\"O}}1 {Ü}{{\"U}}1
	{â}{{\^a}}1 {ê}{{\^e}}1 {î}{{\^i}}1 {ô}{{\^o}}1 {û}{{\^u}}1
	{Â}{{\^A}}1 {Ê}{{\^E}}1 {Î}{{\^I}}1 {Ô}{{\^O}}1 {Û}{{\^U}}1
	{œ}{{\oe}}1 {Œ}{{\OE}}1 {æ}{{\ae}}1 {Æ}{{\AE}}1 {ß}{{\ss}}1
	{ç}{{\c c}}1 {Ç}{{\c C}}1 {ø}{{\o}}1 {å}{{\r a}}1 {Å}{{\r A}}1
	{€}{{\EUR}}1 {£}{{\pounds}}1
}


%\title{Understanding the Code}
%\author{Bendik Steinsvåg Dalen}
\renewcommand\thesubsection{\thesection.\alph{subsection}}
\renewcommand\thesubsubsection{\thesubsection.\roman{subsubsection}}
\begin{document}
	
%[Work in progress].

All of the analysis here is done based on the python file which was copied into each respective folder. However you should keep in mind that I might have copied the wrong file at some point, especially for the earlier ones.

The weights are in the order $\left[ \text{BC}, \text{data}, \text{ODE} \right]$.
 

\section{Fitting all parameters}

\subsection{Baseline}

At first we tried fitting all the parameters at the same time, with exact parameters $a=-0.3$, $b=1.4$, $\tau=20$ and $ I_{\text{ext}}=0.23$. A plot of the exact solution can be seen in figure \ref{plot:exe}.

\begin{figure}[H]
	\centering 
	%Scale angir størrelsen på bildet. Bildefilen må ligge i %samme mappe som tex-filen. 
	\includegraphics[scale=0.7]{../fitzhugh_nagumo_res_nonoise_1e5/plot_exe.pdf}
	\caption{Plot of the exact solution when $a=-0.3$, $b=1.4$, $\tau=20$ and $ I_{\text{ext}}=0.23$}
	%Label gjør det enkelt å referere til ulike bilder.
	\label{plot:exe}
\end{figure}


After $10^5$ epochs we got a prediction that flattened out instead of oscillating, see \ref{plot:all00}. 

\begin{figure}[H]
	\centering 
	%Scale angir størrelsen på bildet. Bildefilen må ligge i %samme mappe som tex-filen. 
	\begin{subfigure}[b]{0.47\textwidth}
		\centering
		\includegraphics[scale=0.43]{../fitzhugh_nagumo_res_nonoise_1e5/plot_pred.pdf}
		\caption{Prediction}
		\label{fig:all00a}
	\end{subfigure}
	\begin{subfigure}[b]{0.47\textwidth}
		\centering
		\includegraphics[scale=0.43]{../fitzhugh_nagumo_res_nonoise_1e5/plot_comp0.pdf}
		\caption{$v$ exact vs. prediction}
		\label{fig:all00b}
	\end{subfigure}
	\begin{subfigure}[b]{0.47\textwidth}
		\centering
		\includegraphics[scale=0.43]{../fitzhugh_nagumo_res_nonoise_1e5/plot_comp1.pdf}
		\caption{$w$ exact vs. prediction}
		\label{fig:all00c}
	\end{subfigure}
	\begin{subfigure}[b]{0.47\textwidth}
		\centering
		\includegraphics[scale=0.43]{../fitzhugh_nagumo_res_nonoise_1e5/plot_pha.pdf}
		\caption{Phase space}
		\label{fig:all00d}
	\end{subfigure}
	\caption{Prediction plots when $a=-0.3$, $b=1.4$, $\tau=20$ and $ I_{\text{ext}}=0.23$ after $10^5$ epochs. With feature transform $t \rightarrow \left[ t, \sin(0.01 \cdot  t), \sin(0.05 \cdot  t), \sin(0.1 \cdot  t), \sin(0.15 \cdot  t)\right] $.}
	%Label gjør det enkelt å referere til ulike bilder.
	\label{plot:all00}
\end{figure}


After $5\cdot10^5$ we got some overflow, which made the plots not appear, see \ref{plot:all01}. For these two first it seems like	I didn't save the original python file for some reason.	

\begin{figure}[H]
	\centering 
	%Scale angir størrelsen på bildet. Bildefilen må ligge i %samme mappe som tex-filen. 
	\begin{subfigure}[b]{0.47\textwidth}
		\centering
		\includegraphics[scale=0.43]{../fitzhugh_nagumo_res_nonoise_5e5/plot_pred.pdf}
		\caption{Prediction}
		\label{fig:all01a}
	\end{subfigure}
	\begin{subfigure}[b]{0.47\textwidth}
		\centering
		\includegraphics[scale=0.43]{../fitzhugh_nagumo_res_nonoise_5e5/plot_comp0.pdf}
		\caption{$v$ exact vs. prediction}
		\label{fig:all01b}
	\end{subfigure}
	\begin{subfigure}[b]{0.47\textwidth}
		\centering
		\includegraphics[scale=0.43]{../fitzhugh_nagumo_res_nonoise_5e5/plot_comp1.pdf}
		\caption{$w$ exact vs. prediction}
		\label{fig:all01c}
	\end{subfigure}
	\begin{subfigure}[b]{0.47\textwidth}
		\centering
		\includegraphics[scale=0.43]{../fitzhugh_nagumo_res_nonoise_5e5/plot_pha.pdf}
		\caption{Phase space}
		\label{fig:all01d}
	\end{subfigure}
	\caption{Prediction plots when $a=-0.3$, $b=1.4$, $\tau=20$ and $ I_{\text{ext}}=0.23$ after $5\cdot10^5$ epochs. With feature transform $t \rightarrow \left[ t, \sin(0.01 \cdot  t), \sin(0.05 \cdot  t), \sin(0.1 \cdot  t), \sin(0.15 \cdot  t)\right] $.}
	%Label gjør det enkelt å referere til ulike bilder.
	\label{plot:all01}
\end{figure} 	
	


In figure \ref{plot:all02} we got better results, but now with only $9\cdot10^4$ epochs. All the weights are set to 1. I'm not sure why we got better results with fewer epochs. It might just be the random state. 

\begin{figure}[H]
	\centering 
	%Scale angir størrelsen på bildet. Bildefilen må ligge i %samme mappe som tex-filen. 
	\begin{subfigure}[b]{0.47\textwidth}
		\centering
		\includegraphics[scale=0.43]{../fitzhugh_nagumo_res0/plot_pred.pdf}
		\caption{Prediction}
		\label{fig:all02a}
	\end{subfigure}
	\begin{subfigure}[b]{0.47\textwidth}
		\centering
		\includegraphics[scale=0.43]{../fitzhugh_nagumo_res0/plot_comp0.pdf}
		\caption{$v$ exact vs. prediction}
		\label{fig:all02b}
	\end{subfigure}
	\begin{subfigure}[b]{0.47\textwidth}
		\centering
		\includegraphics[scale=0.43]{../fitzhugh_nagumo_res0/plot_comp1.pdf}
		\caption{$w$ exact vs. prediction}
		\label{fig:all02c}
	\end{subfigure}
	\begin{subfigure}[b]{0.47\textwidth}
		\centering
		\includegraphics[scale=0.43]{../fitzhugh_nagumo_res0/plot_pha.pdf}
		\caption{Phase space}
		\label{fig:all02d}
	\end{subfigure}
	\caption{Prediction plots when $a=-0.3$, $b=1.4$, $\tau=20$ and $ I_{\text{ext}}=0.23$ after $9\cdot10^4$ epochs. With feature transform $t \rightarrow \left[ t, \sin(0.01 \cdot  t), \sin(0.05 \cdot  t), \sin(0.1 \cdot  t), \sin(0.15 \cdot  t)\right] $, and weights $\left[ \left[ 1, 1\right], \left[ 1, 1\right], \left[ 1, 1\right]\right]$.}
	%Label gjør det enkelt å referere til ulike bilder.
	\label{plot:all02}
\end{figure}



\subsection{Different Scaling}

In figure \ref{plot:all03} we tried having the TensorFlow representation of the variables be scaled using the sigmoid function, instead of the soft plus function. In figure \ref{plot:all04} we tried a Gaussian distribution instead. The saved python file has $\tanh$ here, but I'm pretty sure it used Gaussian.

Other scaling methods were tried as well, including $\tanh$ and no scaling, but these lead to infinities, divide by zero or NaN. 

\begin{figure}[H]
	\centering 
	%Scale angir størrelsen på bildet. Bildefilen må ligge i %samme mappe som tex-filen. 
	\begin{subfigure}[b]{0.47\textwidth}
		\centering
		\includegraphics[scale=0.43]{../fitzhugh_nagumo_res_sigmoid/plot_pred.pdf}
		\caption{Prediction}
		\label{fig:all03a}
	\end{subfigure}
	\begin{subfigure}[b]{0.47\textwidth}
		\centering
		\includegraphics[scale=0.43]{../fitzhugh_nagumo_res_sigmoid/plot_comp0.pdf}
		\caption{$v$ exact vs. prediction}
		\label{fig:all03b}
	\end{subfigure}
	\begin{subfigure}[b]{0.47\textwidth}
		\centering
		\includegraphics[scale=0.43]{../fitzhugh_nagumo_res_sigmoid/plot_comp1.pdf}
		\caption{$w$ exact vs. prediction}
		\label{fig:all03c}
	\end{subfigure}
	\begin{subfigure}[b]{0.47\textwidth}
		\centering
		\includegraphics[scale=0.43]{../fitzhugh_nagumo_res_sigmoid/plot_pha.pdf}
		\caption{Phase space}
		\label{fig:all03d}
	\end{subfigure}
	\caption{Prediction plots when $a=-0.3$, $b=1.4$, $\tau=20$ and $ I_{\text{ext}}=0.23$ after $9\cdot10^4$ epochs. With feature transform $t \rightarrow \left[ t, \sin(0.01 \cdot  t), \sin(0.05 \cdot  t), \sin(0.1 \cdot  t), \sin(0.15 \cdot  t)\right] $, and weights $\left[ \left[ 1, 1\right], \left[ 1, 1\right], \left[ 1, 1\right]\right]$.}
	%Label gjør det enkelt å referere til ulike bilder.
	\label{plot:all03}
\end{figure} 	


\begin{figure}[H]
	\centering 
	%Scale angir størrelsen på bildet. Bildefilen må ligge i %samme mappe som tex-filen. 
	\begin{subfigure}[b]{0.47\textwidth}
		\centering
		\includegraphics[scale=0.43]{../fitzhugh_nagumo_res_gaus/plot_pred.pdf}
		\caption{Prediction}
		\label{fig:all04a}
	\end{subfigure}
	\begin{subfigure}[b]{0.47\textwidth}
		\centering
		\includegraphics[scale=0.43]{../fitzhugh_nagumo_res_gaus/plot_comp0.pdf}
		\caption{$v$ exact vs. prediction}
		\label{fig:all04b}
	\end{subfigure}
	\begin{subfigure}[b]{0.47\textwidth}
		\centering
		\includegraphics[scale=0.43]{../fitzhugh_nagumo_res_gaus/plot_comp1.pdf}
		\caption{$w$ exact vs. prediction}
		\label{fig:all04c}
	\end{subfigure}
	\begin{subfigure}[b]{0.47\textwidth}
		\centering
		\includegraphics[scale=0.43]{../fitzhugh_nagumo_res_gaus/plot_pha.pdf}
		\caption{Phase space}
		\label{fig:all04d}
	\end{subfigure}
	\caption{Prediction plots when $a=-0.3$, $b=1.4$, $\tau=20$ and $ I_{\text{ext}}=0.23$ after $5\cdot10^4$ epochs. With feature transform $t \rightarrow \left[ t, \sin(0.01 \cdot  t), \sin(0.05 \cdot  t), \sin(0.1 \cdot  t), \sin(0.15 \cdot  t)\right] $, and weights $\left[ \left[ 1, 1\right], \left[ 1, 1\right], \left[ 1, 1\right]\right]$.}
	%Label gjør det enkelt å referere til ulike bilder.
	\label{plot:all04}
\end{figure}



\subsubsection{Anchors}

In figure \ref{plot:all05} I tried to have the anchors only use part of \lstinline|data_t| by using the same indices as the randomly selected data. From what I remember it didn't have any noticeable difference, so I think I just used all of it from this point on. However, looking at the plot now it actually looks much worse than figure \ref{plot:all02}. That one was run for more epochs though, which might explain the difference. 

\begin{figure}[H]
	\centering 
	%Scale angir størrelsen på bildet. Bildefilen må ligge i %samme mappe som tex-filen. 
	\begin{subfigure}[b]{0.47\textwidth}
		\centering
		\includegraphics[scale=0.43]{../fitzhugh_nagumo_res_idx/plot_pred.pdf}
		\caption{Prediction}
		\label{fig:all05a}
	\end{subfigure}
	\begin{subfigure}[b]{0.47\textwidth}
		\centering
		\includegraphics[scale=0.43]{../fitzhugh_nagumo_res_idx/plot_comp0.pdf}
		\caption{$v$ exact vs. prediction}
		\label{fig:all05b}
	\end{subfigure}
	\begin{subfigure}[b]{0.47\textwidth}
		\centering
		\includegraphics[scale=0.43]{../fitzhugh_nagumo_res_idx/plot_comp1.pdf}
		\caption{$w$ exact vs. prediction}
		\label{fig:all05c}
	\end{subfigure}
	\begin{subfigure}[b]{0.47\textwidth}
		\centering
		\includegraphics[scale=0.43]{../fitzhugh_nagumo_res_idx/plot_pha.pdf}
		\caption{Phase space}
		\label{fig:all05d}
	\end{subfigure}
	\caption{Prediction plots when $a=-0.3$, $b=1.4$, $\tau=20$ and $ I_{\text{ext}}=0.23$ after $5\cdot10^4$ epochs. With feature transform $t \rightarrow \left[ t, \sin(0.01 \cdot  t), \sin(0.05 \cdot  t), \sin(0.1 \cdot  t), \sin(0.15 \cdot  t)\right] $, and weights $\left[ \left[ 1, 1\right], \left[ 1, 1\right], \left[ 1, 1\right]\right]$.}
	%Label gjør det enkelt å referere til ulike bilder.
	\label{plot:all05}
\end{figure} 	



\subsubsection{No Features}

Figure \ref{plot:all06} shows the result with no feature transform. 

\begin{figure}[H]
	\centering 
	%Scale angir størrelsen på bildet. Bildefilen må ligge i %samme mappe som tex-filen. 
	\begin{subfigure}[b]{0.47\textwidth}
		\centering
		\includegraphics[scale=0.43]{../fitzhugh_nagumo_res_nofeature/plot_pred.pdf}
		\caption{Prediction}
		\label{fig:all06a}
	\end{subfigure}
	\begin{subfigure}[b]{0.47\textwidth}
		\centering
		\includegraphics[scale=0.43]{../fitzhugh_nagumo_res_nofeature/plot_comp0.pdf}
		\caption{$v$ exact vs. prediction}
		\label{fig:all06b}
	\end{subfigure}
	\begin{subfigure}[b]{0.47\textwidth}
		\centering
		\includegraphics[scale=0.43]{../fitzhugh_nagumo_res_nofeature/plot_comp1.pdf}
		\caption{$w$ exact vs. prediction}
		\label{fig:all06c}
	\end{subfigure}
	\begin{subfigure}[b]{0.47\textwidth}
		\centering
		\includegraphics[scale=0.43]{../fitzhugh_nagumo_res_nofeature/plot_pha.pdf}
		\caption{Phase space}
		\label{fig:all06d}
	\end{subfigure}
	\caption{Prediction plots when $a=-0.3$, $b=1.4$, $\tau=20$ and $ I_{\text{ext}}=0.23$ after $5\cdot10^4$ epochs. With no feature transform, and weights $\left[ \left[ 1, 1\right], \left[ 1, 1\right], \left[ 1, 1\right]\right]$.}
	%Label gjør det enkelt å referere til ulike bilder.
	\label{plot:all06}
\end{figure}



\subsubsection{Weights}

Figure \ref{plot:weight01} is the prediction with just the ODE-weights, the rest are set to zero.

\begin{figure}[H]
	\centering 
	%Scale angir størrelsen på bildet. Bildefilen må ligge i %samme mappe som tex-filen. 
	\begin{subfigure}[b]{0.47\textwidth}
		\centering
		\includegraphics[scale=0.43]{../fitzhugh_nagumo_res_justode/plot_pred.pdf}
		\caption{Prediction}
		\label{fig:weight01a}
	\end{subfigure}
	\begin{subfigure}[b]{0.47\textwidth}
		\centering
		\includegraphics[scale=0.43]{../fitzhugh_nagumo_res_justode/plot_comp0.pdf}
		\caption{$v$ exact vs. prediction}
		\label{fig:weight01b}
	\end{subfigure}
	\begin{subfigure}[b]{0.47\textwidth}
		\centering
		\includegraphics[scale=0.43]{../fitzhugh_nagumo_res_justode/plot_comp1.pdf}
		\caption{$w$ exact vs. prediction}
		\label{fig:weight01c}
	\end{subfigure}
	\begin{subfigure}[b]{0.47\textwidth}
		\centering
		\includegraphics[scale=0.43]{../fitzhugh_nagumo_res_justode/plot_pha.pdf}
		\caption{Phase space}
		\label{fig:weight01d}
	\end{subfigure}
	\caption{Prediction plots when $a=-0.3$, $b=1.4$, $\tau=20$ and $ I_{\text{ext}}=0.23$ after $5\cdot10^4$ epochs. With feature transform $t \rightarrow \left[ t, \sin(0.01 \cdot  t), \sin(0.05 \cdot  t), \sin(0.1 \cdot  t), \sin(0.15 \cdot  t)\right] $, and weights $\left[ \left[ 0, 0\right], \left[ 0, 0\right], \left[ 1, 1\right]\right]$.}
	%Label gjør det enkelt å referere til ulike bilder.
	\label{plot:weight01}
\end{figure} 	

Figure \ref{plot:weight02} is the prediction with just the data-weights, the rest are set to zero.

\begin{figure}[H]
	\centering 
	%Scale angir størrelsen på bildet. Bildefilen må ligge i %samme mappe som tex-filen. 
	\begin{subfigure}[b]{0.47\textwidth}
		\centering
		\includegraphics[scale=0.43]{../fitzhugh_nagumo_res_justdata/plot_pred.pdf}
		\caption{Prediction}
		\label{fig:weight02a}
	\end{subfigure}
	\begin{subfigure}[b]{0.47\textwidth}
		\centering
		\includegraphics[scale=0.43]{../fitzhugh_nagumo_res_justdata/plot_comp0.pdf}
		\caption{$v$ exact vs. prediction}
		\label{fig:weight02b}
	\end{subfigure}
	\begin{subfigure}[b]{0.47\textwidth}
		\centering
		\includegraphics[scale=0.43]{../fitzhugh_nagumo_res_justdata/plot_comp1.pdf}
		\caption{$w$ exact vs. prediction}
		\label{fig:weight02c}
	\end{subfigure}
	\begin{subfigure}[b]{0.47\textwidth}
		\centering
		\includegraphics[scale=0.43]{../fitzhugh_nagumo_res_justdata/plot_pha.pdf}
		\caption{Phase space}
		\label{fig:weight02d}
	\end{subfigure}
	\caption{Prediction plots when $a=-0.3$, $b=1.4$, $\tau=20$ and $ I_{\text{ext}}=0.23$ after $5\cdot10^4$ epochs. With feature transform $t \rightarrow \left[ t, \sin(0.01 \cdot  t), \sin(0.05 \cdot  t), \sin(0.1 \cdot  t), \sin(0.15 \cdot  t)\right] $, and weights $\left[ \left[ 0, 0\right], \left[ 1, 1\right], \left[ 0, 0\right]\right]$.}
	%Label gjør det enkelt å referere til ulike bilder.
	\label{plot:weight02}
\end{figure}

Figure \ref{plot:weight03} is the prediction with just the auxiliary-weights, the rest are set to zero.

\begin{figure}[H]
	\centering 
	%Scale angir størrelsen på bildet. Bildefilen må ligge i %samme mappe som tex-filen. 
	\begin{subfigure}[b]{0.47\textwidth}
		\centering
		\includegraphics[scale=0.43]{../fitzhugh_nagumo_res_justaux/plot_pred.pdf}
		\caption{Prediction}
		\label{fig:weight03a}
	\end{subfigure}
	\begin{subfigure}[b]{0.47\textwidth}
		\centering
		\includegraphics[scale=0.43]{../fitzhugh_nagumo_res_justaux/plot_comp0.pdf}
		\caption{$v$ exact vs. prediction}
		\label{fig:weight03b}
	\end{subfigure}
	\begin{subfigure}[b]{0.47\textwidth}
		\centering
		\includegraphics[scale=0.43]{../fitzhugh_nagumo_res_justaux/plot_comp1.pdf}
		\caption{$w$ exact vs. prediction}
		\label{fig:weight03c}
	\end{subfigure}
	\begin{subfigure}[b]{0.47\textwidth}
		\centering
		\includegraphics[scale=0.43]{../fitzhugh_nagumo_res_justaux/plot_pha.pdf}
		\caption{Phase space}
		\label{fig:weight03d}
	\end{subfigure}
	\caption{Prediction plots when $a=-0.3$, $b=1.4$, $\tau=20$ and $ I_{\text{ext}}=0.23$ after $9\cdot10^4$ epochs. With feature transform $t \rightarrow \left[ t, \sin(0.01 \cdot  t), \sin(0.05 \cdot  t), \sin(0.1 \cdot  t), \sin(0.15 \cdot  t)\right] $, and weights $\left[ \left[ 1, 1\right], \left[ 0, 0\right], \left[ 0, 0\right]\right]$.}
	%Label gjør det enkelt å referere til ulike bilder.
	\label{plot:weight03}
\end{figure} 	

Figure \ref{plot:weight04} is the prediction with all weights set to zero.

\begin{figure}[H]
	\centering 
	%Scale angir størrelsen på bildet. Bildefilen må ligge i %samme mappe som tex-filen. 
	\begin{subfigure}[b]{0.47\textwidth}
		\centering
		\includegraphics[scale=0.43]{../fitzhugh_nagumo_res_zeroweights/plot_pred.pdf}
		\caption{Prediction}
		\label{fig:weight04a}
	\end{subfigure}
	\begin{subfigure}[b]{0.47\textwidth}
		\centering
		\includegraphics[scale=0.43]{../fitzhugh_nagumo_res_zeroweights/plot_comp0.pdf}
		\caption{$v$ exact vs. prediction}
		\label{fig:weight04b}
	\end{subfigure}
	\begin{subfigure}[b]{0.47\textwidth}
		\centering
		\includegraphics[scale=0.43]{../fitzhugh_nagumo_res_zeroweights/plot_comp1.pdf}
		\caption{$w$ exact vs. prediction}
		\label{fig:weight04c}
	\end{subfigure}
	\begin{subfigure}[b]{0.47\textwidth}
		\centering
		\includegraphics[scale=0.43]{../fitzhugh_nagumo_res_zeroweights/plot_pha.pdf}
		\caption{Phase space}
		\label{fig:weight04d}
	\end{subfigure}
	\caption{Prediction plots when $a=-0.3$, $b=1.4$, $\tau=20$ and $ I_{\text{ext}}=0.23$ after $5\cdot10^4$ epochs. With feature transform $t \rightarrow \left[ t, \sin(0.01 \cdot  t), \sin(0.05 \cdot  t), \sin(0.1 \cdot  t), \sin(0.15 \cdot  t)\right] $, and weights $\left[ \left[ 0, 0\right], \left[ 0, 0\right], \left[ 0, 0\right]\right]$.}
	%Label gjør det enkelt å referere til ulike bilder.
	\label{plot:weight04}
\end{figure}


Figure \ref{plot:weight05} is the prediction with the ODE-weights set to zero, the rest are 1.

\begin{figure}[H]
	\centering 
	%Scale angir størrelsen på bildet. Bildefilen må ligge i %samme mappe som tex-filen. 
	\begin{subfigure}[b]{0.47\textwidth}
		\centering
		\includegraphics[scale=0.43]{../fitzhugh_nagumo_res_noode/plot_pred.pdf}
		\caption{Prediction}
		\label{fig:weight05a}
	\end{subfigure}
	\begin{subfigure}[b]{0.47\textwidth}
		\centering
		\includegraphics[scale=0.43]{../fitzhugh_nagumo_res_noode/plot_comp0.pdf}
		\caption{$v$ exact vs. prediction}
		\label{fig:weight05b}
	\end{subfigure}
	\begin{subfigure}[b]{0.47\textwidth}
		\centering
		\includegraphics[scale=0.43]{../fitzhugh_nagumo_res_noode/plot_comp1.pdf}
		\caption{$w$ exact vs. prediction}
		\label{fig:weight05c}
	\end{subfigure}
	\begin{subfigure}[b]{0.47\textwidth}
		\centering
		\includegraphics[scale=0.43]{../fitzhugh_nagumo_res_noode/plot_pha.pdf}
		\caption{Phase space}
		\label{fig:weight05d}
	\end{subfigure}
	\caption{Prediction plots when $a=-0.3$, $b=1.4$, $\tau=20$ and $ I_{\text{ext}}=0.23$ after $5\cdot10^4$ epochs. With feature transform $t \rightarrow \left[ t, \sin(0.01 \cdot  t), \sin(0.05 \cdot  t), \sin(0.1 \cdot  t), \sin(0.15 \cdot  t)\right] $, and weights $\left[ \left[ 1, 1\right], \left[ 1, 1\right], \left[ 0, 0\right]\right]$.}
	%Label gjør det enkelt å referere til ulike bilder.
	\label{plot:weight05}
\end{figure} 	

Figure \ref{plot:weight06} is the prediction with the data-weights set to zero, the rest are 1.

\begin{figure}[H]
	\centering 
	%Scale angir størrelsen på bildet. Bildefilen må ligge i %samme mappe som tex-filen. 
	\begin{subfigure}[b]{0.47\textwidth}
		\centering
		\includegraphics[scale=0.43]{../fitzhugh_nagumo_res_nodata/plot_pred.pdf}
		\caption{Prediction}
		\label{fig:weight06a}
	\end{subfigure}
	\begin{subfigure}[b]{0.47\textwidth}
		\centering
		\includegraphics[scale=0.43]{../fitzhugh_nagumo_res_nodata/plot_comp0.pdf}
		\caption{$v$ exact vs. prediction}
		\label{fig:weight06b}
	\end{subfigure}
	\begin{subfigure}[b]{0.47\textwidth}
		\centering
		\includegraphics[scale=0.43]{../fitzhugh_nagumo_res_nodata/plot_comp1.pdf}
		\caption{$w$ exact vs. prediction}
		\label{fig:weight06c}
	\end{subfigure}
	\begin{subfigure}[b]{0.47\textwidth}
		\centering
		\includegraphics[scale=0.43]{../fitzhugh_nagumo_res_nodata/plot_pha.pdf}
		\caption{Phase space}
		\label{fig:weight06d}
	\end{subfigure}
	\caption{Prediction plots when $a=-0.3$, $b=1.4$, $\tau=20$ and $ I_{\text{ext}}=0.23$ after $5\cdot10^4$ epochs. With feature transform $t \rightarrow \left[ t, \sin(0.01 \cdot  t), \sin(0.05 \cdot  t), \sin(0.1 \cdot  t), \sin(0.15 \cdot  t)\right] $, and weights $\left[ \left[ 1, 1\right], \left[ 0, 0\right], \left[ 1, 1\right]\right]$.}
	%Label gjør det enkelt å referere til ulike bilder.
	\label{plot:weight06}
\end{figure}

Figure \ref{plot:weight07} is the prediction with the auxiliary-weights set to zero, the rest are 1.

\begin{figure}[H]
	\centering 
	%Scale angir størrelsen på bildet. Bildefilen må ligge i %samme mappe som tex-filen. 
	\begin{subfigure}[b]{0.47\textwidth}
		\centering
		\includegraphics[scale=0.43]{../fitzhugh_nagumo_res_noaux/plot_pred.pdf}
		\caption{Prediction}
		\label{fig:weight07a}
	\end{subfigure}
	\begin{subfigure}[b]{0.47\textwidth}
		\centering
		\includegraphics[scale=0.43]{../fitzhugh_nagumo_res_noaux/plot_comp0.pdf}
		\caption{$v$ exact vs. prediction}
		\label{fig:weight07b}
	\end{subfigure}
	\begin{subfigure}[b]{0.47\textwidth}
		\centering
		\includegraphics[scale=0.43]{../fitzhugh_nagumo_res_noaux/plot_comp1.pdf}
		\caption{$w$ exact vs. prediction}
		\label{fig:weight07c}
	\end{subfigure}
	\begin{subfigure}[b]{0.47\textwidth}
		\centering
		\includegraphics[scale=0.43]{../fitzhugh_nagumo_res_noaux/plot_pha.pdf}
		\caption{Phase space}
		\label{fig:weight07d}
	\end{subfigure}
	\caption{Prediction plots when $a=-0.3$, $b=1.4$, $\tau=20$ and $ I_{\text{ext}}=0.23$ after $9\cdot10^4$ epochs. With feature transform $t \rightarrow \left[ t, \sin(0.01 \cdot  t), \sin(0.05 \cdot  t), \sin(0.1 \cdot  t), \sin(0.15 \cdot  t)\right] $, and weights $\left[ \left[ 0, 0\right], \left[ 1, 1\right], \left[ 1, 1\right]\right]$.}
	%Label gjør det enkelt å referere til ulike bilder.
	\label{plot:weight07}
\end{figure} 	


\subsection{More Baseline}



\begin{figure}[H]
	\centering 
	%Scale angir størrelsen på bildet. Bildefilen må ligge i %samme mappe som tex-filen. 
	\begin{subfigure}[b]{0.47\textwidth}
		\centering
		\includegraphics[scale=0.43]{../fitzhugh_nagumo_res1/plot_pred.pdf}
		\caption{Prediction}
		\label{fig:all11a}
	\end{subfigure}
	\begin{subfigure}[b]{0.47\textwidth}
		\centering
		\includegraphics[scale=0.43]{../fitzhugh_nagumo_res1/plot_comp0.pdf}
		\caption{$v$ exact vs. prediction}
		\label{fig:all11b}
	\end{subfigure}
	\begin{subfigure}[b]{0.47\textwidth}
		\centering
		\includegraphics[scale=0.43]{../fitzhugh_nagumo_res1/plot_comp1.pdf}
		\caption{$w$ exact vs. prediction}
		\label{fig:all11c}
	\end{subfigure}
	\begin{subfigure}[b]{0.47\textwidth}
		\centering
		\includegraphics[scale=0.43]{../fitzhugh_nagumo_res1/plot_pha.pdf}
		\caption{Phase space}
		\label{fig:all11d}
	\end{subfigure}
	\caption{Prediction plots when $a=-0.3$, $b=1.4$, $\tau=20$ and $ I_{\text{ext}}=0.23$ after $9\cdot10^4$ epochs. With feature transform $t \rightarrow \left[ t, \sin(0.01 \cdot  t), \sin(0.05 \cdot  t), \sin(0.1 \cdot  t), \sin(0.15 \cdot  t)\right] $, and weights $\left[ \left[ 1, 1\right], \left[ 1, 1\right], \left[ 1, 1\right]\right]$.}
	%Label gjør det enkelt å referere til ulike bilder.
	\label{plot:all11}
\end{figure}


\subsection{Negative $a$}

In figure \ref{plot:neg1} and \ref{plot:neg2} I tried forcing $a$ to be negative by setting the scaling multiplier to be negative.

\begin{figure}[H]
	\centering 
	%Scale angir størrelsen på bildet. Bildefilen må ligge i %samme mappe som tex-filen. 
	\begin{subfigure}[b]{0.47\textwidth}
		\centering
		\includegraphics[scale=0.43]{../fitzhugh_nagumo_res_neg/plot_pred.pdf}
		\caption{Prediction}
		\label{fig:neg1a}
	\end{subfigure}
	\begin{subfigure}[b]{0.47\textwidth}
		\centering
		\includegraphics[scale=0.43]{../fitzhugh_nagumo_res_neg/plot_comp0.pdf}
		\caption{$v$ exact vs. prediction}
		\label{fig:neg1b}
	\end{subfigure}
	\begin{subfigure}[b]{0.47\textwidth}
		\centering
		\includegraphics[scale=0.43]{../fitzhugh_nagumo_res_neg/plot_comp1.pdf}
		\caption{$w$ exact vs. prediction}
		\label{fig:neg1c}
	\end{subfigure}
	\begin{subfigure}[b]{0.47\textwidth}
		\centering
		\includegraphics[scale=0.43]{../fitzhugh_nagumo_res_neg/plot_pha.pdf}
		\caption{Phase space}
		\label{fig:neg1d}
	\end{subfigure}
	\caption{Prediction plots when $a=-0.3$, $b=1.4$, $\tau=20$ and $ I_{\text{ext}}=0.23$ after $9\cdot10^4$ epochs. With feature transform $t \rightarrow \left[ t, \sin(0.01 \cdot  t), \sin(0.05 \cdot  t), \sin(0.1 \cdot  t), \sin(0.15 \cdot  t)\right] $, and weights $\left[ \left[ 1, 1\right], \left[ 1, 1\right], \left[ 1, 1\right]\right]$.}
	%Label gjør det enkelt å referere til ulike bilder.
	\label{plot:neg1}
\end{figure}

\begin{figure}[H]
	\centering 
	%Scale angir størrelsen på bildet. Bildefilen må ligge i %samme mappe som tex-filen. 
	\begin{subfigure}[b]{0.47\textwidth}
		\centering
		\includegraphics[scale=0.43]{../fitzhugh_nagumo_res_neg0/plot_pred.pdf}
		\caption{Prediction}
		\label{fig:neg2a}
	\end{subfigure}
	\begin{subfigure}[b]{0.47\textwidth}
		\centering
		\includegraphics[scale=0.43]{../fitzhugh_nagumo_res_neg0/plot_comp0.pdf}
		\caption{$v$ exact vs. prediction}
		\label{fig:neg2b}
	\end{subfigure}
	\begin{subfigure}[b]{0.47\textwidth}
		\centering
		\includegraphics[scale=0.43]{../fitzhugh_nagumo_res_neg0/plot_comp1.pdf}
		\caption{$w$ exact vs. prediction}
		\label{fig:neg2c}
	\end{subfigure}
	\begin{subfigure}[b]{0.47\textwidth}
		\centering
		\includegraphics[scale=0.43]{../fitzhugh_nagumo_res_neg0/plot_pha.pdf}
		\caption{Phase space}
		\label{fig:neg2d}
	\end{subfigure}
	\caption{Prediction plots when $a=-0.3$, $b=1.4$, $\tau=20$ and $ I_{\text{ext}}=0.23$ after $10^5$ epochs. With feature transform $t \rightarrow \left[ t, \sin(0.01 \cdot  t), \sin(0.05 \cdot  t), \sin(0.1 \cdot  t), \sin(0.15 \cdot  t)\right] $, and weights $\left[ \left[ 1, 1\right], \left[ 1, 1\right], \left[ 1, 1\right]\right]$.}
	%Label gjør det enkelt å referere til ulike bilder.
	\label{plot:neg2}
\end{figure}



\subsection{Changing feature transform}

Here we tried changing the input feature transformation. In figure \ref{plot:feature00} the feature transformation was $t \rightarrow \left[ t, \sin(0.013\cdot t) \right] $. We observed that the model had 13 peaks in 1000$ms$, and tried to mimic that. For the first attempt here I forgot to include $2\pi$ to get the proper periodicity. 

\begin{figure}[H]
	\centering 
	%Scale angir størrelsen på bildet. Bildefilen må ligge i %samme mappe som tex-filen. 
	\begin{subfigure}[b]{0.47\textwidth}
		\centering
		\includegraphics[scale=0.43]{../fitzhugh_nagumo_res_feature0/plot_pred.pdf}
		\caption{Prediction}
		\label{fig:feature00a}
	\end{subfigure}
	\begin{subfigure}[b]{0.47\textwidth}
		\centering
		\includegraphics[scale=0.43]{../fitzhugh_nagumo_res_feature0/plot_comp0.pdf}
		\caption{$v$ exact vs. prediction}
		\label{fig:feature00b}
	\end{subfigure}
	\begin{subfigure}[b]{0.47\textwidth}
		\centering
		\includegraphics[scale=0.43]{../fitzhugh_nagumo_res_feature0/plot_comp1.pdf}
		\caption{$w$ exact vs. prediction}
		\label{fig:feature00c}
	\end{subfigure}
	\begin{subfigure}[b]{0.47\textwidth}
		\centering
		\includegraphics[scale=0.43]{../fitzhugh_nagumo_res_feature0/plot_pha.pdf}
		\caption{Phase space}
		\label{fig:feature00d}
	\end{subfigure}
	\caption{Prediction plots when $a=-0.3$, $b=1.4$, $\tau=20$ and $ I_{\text{ext}}=0.23$ after $10^5$ epochs. With feature transform $t \rightarrow \left[ t, \sin(0.013\cdot t) \right] $, and weights $\left[ \left[ 1, 1\right], \left[ 1, 1\right], \left[ 1, 1\right]\right]$.}
	%Label gjør det enkelt å referere til ulike bilder.
	\label{plot:feature00}
\end{figure} 	

In figure \ref{plot:feature01} the feature transformation was $t \rightarrow \left[t, \sin(0.013\cdot 2 \pi t) \right]$. 

\begin{figure}[H]
	\centering 
	%Scale angir størrelsen på bildet. Bildefilen må ligge i %samme mappe som tex-filen. 
	\begin{subfigure}[b]{0.47\textwidth}
		\centering
		\includegraphics[scale=0.43]{../fitzhugh_nagumo_res_feature1/plot_pred.pdf}
		\caption{Prediction}
		\label{fig:feature01a}
	\end{subfigure}
	\begin{subfigure}[b]{0.47\textwidth}
		\centering
		\includegraphics[scale=0.43]{../fitzhugh_nagumo_res_feature1/plot_comp0.pdf}
		\caption{$v$ exact vs. prediction}
		\label{fig:feature01b}
	\end{subfigure}
	\begin{subfigure}[b]{0.47\textwidth}
		\centering
		\includegraphics[scale=0.43]{../fitzhugh_nagumo_res_feature1/plot_comp1.pdf}
		\caption{$w$ exact vs. prediction}
		\label{fig:feature01c}
	\end{subfigure}
	\begin{subfigure}[b]{0.47\textwidth}
		\centering
		\includegraphics[scale=0.43]{../fitzhugh_nagumo_res_feature1/plot_pha.pdf}
		\caption{Phase space}
		\label{fig:feature01d}
	\end{subfigure}
	\caption{Prediction plots when $a=-0.3$, $b=1.4$, $\tau=20$ and $ I_{\text{ext}}=0.23$ after $10^5$ epochs. With feature transform $t \rightarrow \left[  t, \sin(0.013\cdot 2 \pi t) \right]$, and weights $\left[ \left[ 1, 1\right], \left[ 1, 1\right], \left[ 1, 1\right]\right]$.}
	%Label gjør det enkelt å referere til ulike bilder.
	\label{plot:feature01}
\end{figure}

In figure \ref{plot:feature02}  the feature transformation was $t \rightarrow \left[  t, \sin(0.005\cdot 2 \pi t), \sin(0.01\cdot 2 \pi t), \sin(0.013\cdot 2 \pi t), \sin(0.02\cdot 2 \pi t) \right]$. 

\begin{figure}[H]
	\centering 
	%Scale angir størrelsen på bildet. Bildefilen må ligge i %samme mappe som tex-filen. 
	\begin{subfigure}[b]{0.47\textwidth}
		\centering
		\includegraphics[scale=0.43]{../fitzhugh_nagumo_res_feature2/plot_pred.pdf}
		\caption{Prediction}
		\label{fig:feature02a}
	\end{subfigure}
	\begin{subfigure}[b]{0.47\textwidth}
		\centering
		\includegraphics[scale=0.43]{../fitzhugh_nagumo_res_feature2/plot_comp0.pdf}
		\caption{$v$ exact vs. prediction}
		\label{fig:feature02b}
	\end{subfigure}
	\begin{subfigure}[b]{0.47\textwidth}
		\centering
		\includegraphics[scale=0.43]{../fitzhugh_nagumo_res_feature2/plot_comp1.pdf}
		\caption{$w$ exact vs. prediction}
		\label{fig:feature02c}
	\end{subfigure}
	\begin{subfigure}[b]{0.47\textwidth}
		\centering
		\includegraphics[scale=0.43]{../fitzhugh_nagumo_res_feature2/plot_pha.pdf}
		\caption{Phase space}
		\label{fig:feature02d}
	\end{subfigure}
	\caption{Prediction plots when $a=-0.3$, $b=1.4$, $\tau=20$ and $ I_{\text{ext}}=0.23$ after $10^5$ epochs. With feature transform $t \rightarrow \left[  t, \sin(0.005\cdot 2 \pi t), \sin(0.01\cdot 2 \pi t), \sin(0.013\cdot 2 \pi t), \sin(0.02\cdot 2 \pi t) \right]$, and weights $\left[ \left[ 1, 1\right], \left[ 1, 1\right], \left[ 1, 1\right]\right]$.}
	%Label gjør det enkelt å referere til ulike bilder.
	\label{plot:feature02}
\end{figure} 	


In figure \ref{plot:feature03} the feature transformation was $t \rightarrow \left[t, \sin(0.013\cdot 2 \pi t) \right]$. 

\begin{figure}[H]
	\centering 
	%Scale angir størrelsen på bildet. Bildefilen må ligge i %samme mappe som tex-filen. 
	\begin{subfigure}[b]{0.47\textwidth}
		\centering
		\includegraphics[scale=0.43]{../fitzhugh_nagumo_res_feature3/plot_pred.pdf}
		\caption{Prediction}
		\label{fig:feature03a}
	\end{subfigure}
	\begin{subfigure}[b]{0.47\textwidth}
		\centering
		\includegraphics[scale=0.43]{../fitzhugh_nagumo_res_feature3/plot_comp0.pdf}
		\caption{$v$ exact vs. prediction}
		\label{fig:feature03b}
	\end{subfigure}
	\begin{subfigure}[b]{0.47\textwidth}
		\centering
		\includegraphics[scale=0.43]{../fitzhugh_nagumo_res_feature3/plot_comp1.pdf}
		\caption{$w$ exact vs. prediction}
		\label{fig:feature03c}
	\end{subfigure}
	\begin{subfigure}[b]{0.47\textwidth}
		\centering
		\includegraphics[scale=0.43]{../fitzhugh_nagumo_res_feature3/plot_pha.pdf}
		\caption{Phase space}
		\label{fig:feature03d}
	\end{subfigure}
	\caption{Prediction plots when $a=-0.3$, $b=1.4$, $\tau=20$ and $ I_{\text{ext}}=0.23$ after $3\cdot10^5$ epochs. With feature transform $t \rightarrow \left[  t, \sin(0.013\cdot 2 \pi t) \right]$, and weights $\left[ \left[ 1, 1\right], \left[ 1, 1\right], \left[ 1, 1\right]\right]$.}
	%Label gjør det enkelt å referere til ulike bilder.
	\label{plot:feature03}
\end{figure}


In figure \ref{plot:feature04}  the feature transformation was $t \rightarrow \left[  t, \sin(0.005\cdot 2 \pi t), \sin(0.01\cdot 2 \pi t), \sin(0.015\cdot 2 \pi t), \sin(0.02\cdot 2 \pi t) \right]$. 

\begin{figure}[H]
	\centering 
	%Scale angir størrelsen på bildet. Bildefilen må ligge i %samme mappe som tex-filen. 
	\begin{subfigure}[b]{0.47\textwidth}
		\centering
		\includegraphics[scale=0.43]{../fitzhugh_nagumo_res2/plot_pred.pdf}
		\caption{Prediction}
		\label{fig:feature04a}
	\end{subfigure}
	\begin{subfigure}[b]{0.47\textwidth}
		\centering
		\includegraphics[scale=0.43]{../fitzhugh_nagumo_res2/plot_comp0.pdf}
		\caption{$v$ exact vs. prediction}
		\label{fig:feature04b}
	\end{subfigure}
	\begin{subfigure}[b]{0.47\textwidth}
		\centering
		\includegraphics[scale=0.43]{../fitzhugh_nagumo_res2/plot_comp1.pdf}
		\caption{$w$ exact vs. prediction}
		\label{fig:feature04c}
	\end{subfigure}
	\begin{subfigure}[b]{0.47\textwidth}
		\centering
		\includegraphics[scale=0.43]{../fitzhugh_nagumo_res2/plot_pha.pdf}
		\caption{Phase space}
		\label{fig:feature04d}
	\end{subfigure}
	\caption{Prediction plots when $a=-0.3$, $b=1.4$, $\tau=20$ and $ I_{\text{ext}}=0.23$ after $10^5$ epochs. With feature transform $t \rightarrow \left[  t, \sin(0.005\cdot 2 \pi t), \sin(0.01\cdot 2 \pi t), \sin(0.015\cdot 2 \pi t), \sin(0.02\cdot 2 \pi t) \right]$, and weights $\left[ \left[ 1, 1\right], \left[ 10^3, 1\right], \left[ 10^{-2}, 10^{-2}\right]\right]$.}
	%Label gjør det enkelt å referere til ulike bilder.
	\label{plot:feature04}
\end{figure}


\section{Fitting $a$ and $b$}

In figure \ref{plot:ab0} and \ref{plot:ab1} we kept $\tau$ and $I_{\text{ext}}$ constant, and tried to only fit $a$ and $b$.
%something about the strange date? I think the result is the correct one, but the date when the results were gathered might be wrong. 

\begin{figure}[H]
	\centering 
	%Scale angir størrelsen på bildet. Bildefilen må ligge i %samme mappe som tex-filen. 
	\begin{subfigure}[b]{0.47\textwidth}
		\centering
		\includegraphics[scale=0.43]{../fitzhugh_nagumo_res_all_tIconst/plot_pred.pdf}
		\caption{Prediction}
		\label{fig:ab0a}
	\end{subfigure}
	\begin{subfigure}[b]{0.47\textwidth}
		\centering
		\includegraphics[scale=0.43]{../fitzhugh_nagumo_res_all_tIconst/plot_comp0.pdf}
		\caption{$v$ exact vs. prediction}
		\label{fig:ab0b}
	\end{subfigure}
	\begin{subfigure}[b]{0.47\textwidth}
		\centering
		\includegraphics[scale=0.43]{../fitzhugh_nagumo_res_all_tIconst/plot_comp1.pdf}
		\caption{$w$ exact vs. prediction}
		\label{fig:ab0c}
	\end{subfigure}
	\begin{subfigure}[b]{0.47\textwidth}
		\centering
		\includegraphics[scale=0.43]{../fitzhugh_nagumo_res_all_tIconst/plot_pha.pdf}
		\caption{Phase space}
		\label{fig:ab0d}
	\end{subfigure}
	\caption{Prediction plots when $a=-0.3$, $b=1.4$, $\tau=20$ and $ I_{\text{ext}}=0.23$ after $5\cdot10^4$ epochs. With feature transform $t \rightarrow \left[  t, \sin(0.013\cdot 2 \pi t) \right] $, and weights $\left[ \left[ 1, 1\right], \left[ 1, 1\right], \left[ 1, 1\right]\right]$.}
	%Label gjør det enkelt å referere til ulike bilder.
	\label{plot:ab0}
\end{figure}


\begin{figure}[H]
	\centering 
	%Scale angir størrelsen på bildet. Bildefilen må ligge i %samme mappe som tex-filen. 
	\begin{subfigure}[b]{0.47\textwidth}
		\centering
		\includegraphics[scale=0.43]{../fitzhugh_nagumo_res_feature_onlyb/plot_pred.pdf}
		\caption{Prediction}
		\label{fig:ab1a}
	\end{subfigure}
	\begin{subfigure}[b]{0.47\textwidth}
		\centering
		\includegraphics[scale=0.43]{../fitzhugh_nagumo_res_feature_onlyb/plot_comp0.pdf}
		\caption{$v$ exact vs. prediction}
		\label{fig:ab1b}
	\end{subfigure}
	\begin{subfigure}[b]{0.47\textwidth}
		\centering
		\includegraphics[scale=0.43]{../fitzhugh_nagumo_res_feature_onlyb/plot_comp1.pdf}
		\caption{$w$ exact vs. prediction}
		\label{fig:ab1c}
	\end{subfigure}
	\begin{subfigure}[b]{0.47\textwidth}
		\centering
		\includegraphics[scale=0.43]{../fitzhugh_nagumo_res_feature_onlyb/plot_pha.pdf}
		\caption{Phase space}
		\label{fig:ab1d}
	\end{subfigure}
	\caption{Prediction plots when $a=-0.3$, $b=1.4$, $\tau=20$ and $ I_{\text{ext}}=0.23$ after $5\cdot10^4$ epochs. With feature transform $t \rightarrow \left[ t, \sin(0.013\cdot t) \right] $, and weights $\left[ \left[ 1, 1\right], \left[ 1, 1\right], \left[ 1, 1\right]\right]$.}
	%Label gjør det enkelt å referere til ulike bilder.
	\label{plot:ab1}
\end{figure} 	


\section{Fitting only $b$}

\subsection{Baseline}

In this section we tried to only fit $b$ while keeping the other variables constant. 

\begin{figure}[H]
	\centering 
	%Scale angir størrelsen på bildet. Bildefilen må ligge i %samme mappe som tex-filen. 
	\begin{subfigure}[b]{0.47\textwidth}
		\centering
		\includegraphics[scale=0.43]{../fitzhugh_nagumo_res_feature_onlyb_lowode/plot_pred.pdf}
		\caption{Prediction}
		\label{fig:justb00a}
	\end{subfigure}
	\begin{subfigure}[b]{0.47\textwidth}
		\centering
		\includegraphics[scale=0.43]{../fitzhugh_nagumo_res_feature_onlyb_lowode/plot_comp0.pdf}
		\caption{$v$ exact vs. prediction}
		\label{fig:justb00b}
	\end{subfigure}
	\begin{subfigure}[b]{0.47\textwidth}
		\centering
		\includegraphics[scale=0.43]{../fitzhugh_nagumo_res_feature_onlyb_lowode/plot_comp1.pdf}
		\caption{$w$ exact vs. prediction}
		\label{fig:justb00c}
	\end{subfigure}
	\begin{subfigure}[b]{0.47\textwidth}
		\centering
		\includegraphics[scale=0.43]{../fitzhugh_nagumo_res_feature_onlyb_lowode/plot_pha.pdf}
		\caption{Phase space}
		\label{fig:justb00d}
	\end{subfigure}
	\caption{Prediction plots when $a=-0.3$, $b=1.4$, $\tau=20$ and $ I_{\text{ext}}=0.23$ after $10^5$ epochs. With feature transform $t \rightarrow \left[ t, \sin(0.013\cdot t) \right] $, and weights $\left[ \left[ 1, 1\right], \left[ 10^3, 1\right], \left[ 10^{-2}, 10^{-2}\right]\right]$.}
	%Label gjør det enkelt å referere til ulike bilder.
	\label{plot:justb00}
\end{figure} 	


\begin{figure}[H]
	\centering 
	%Scale angir størrelsen på bildet. Bildefilen må ligge i %samme mappe som tex-filen. 
	\begin{subfigure}[b]{0.47\textwidth}
		\centering
		\includegraphics[scale=0.43]{../fitzhugh_nagumo_res_feature_onlyb_0/plot_pred.pdf}
		\caption{Prediction}
		\label{fig:justb01a}
	\end{subfigure}
	\begin{subfigure}[b]{0.47\textwidth}
		\centering
		\includegraphics[scale=0.43]{../fitzhugh_nagumo_res_feature_onlyb_0/plot_comp0.pdf}
		\caption{$v$ exact vs. prediction}
		\label{fig:justb01b}
	\end{subfigure}
	\begin{subfigure}[b]{0.47\textwidth}
		\centering
		\includegraphics[scale=0.43]{../fitzhugh_nagumo_res_feature_onlyb_0/plot_comp1.pdf}
		\caption{$w$ exact vs. prediction}
		\label{fig:justb01c}
	\end{subfigure}
	\begin{subfigure}[b]{0.47\textwidth}
		\centering
		\includegraphics[scale=0.43]{../fitzhugh_nagumo_res_feature_onlyb_0/plot_pha.pdf}
		\caption{Phase space}
		\label{fig:justb01d}
	\end{subfigure}
	\caption{Prediction plots when $a=-0.3$, $b=1.4$, $\tau=20$ and $ I_{\text{ext}}=0.23$ after $10^5$ epochs. With feature transform $t \rightarrow \left[ t, \sin(0.013\cdot t) \right] $, and weights $\left[ \left[ 1, 1\right], \left[ 10, 10\right], \left[ 10^{-2}, 10^{-2}\right]\right]$.}
	%Label gjør det enkelt å referere til ulike bilder.
	\label{plot:justb01}
\end{figure} 	

We realised that the FHN-model is quite unstable. For example, a value of $b>1.43$ will give a flat curve. 
In figure \ref{plot:justb02} we tried setting the real value of $b$ to $1.1$ instead. %because instability

\begin{figure}[H]
	\centering 
	%Scale angir størrelsen på bildet. Bildefilen må ligge i %samme mappe som tex-filen. 
	\begin{subfigure}[b]{0.47\textwidth}
		\centering
		\includegraphics[scale=0.43]{../fitzhugh_nagumo_res_feature_onlyb_1/plot_pred.pdf}
		\caption{Prediction}
		\label{fig:justb02a}
	\end{subfigure}
	\begin{subfigure}[b]{0.47\textwidth}
		\centering
		\includegraphics[scale=0.43]{../fitzhugh_nagumo_res_feature_onlyb_1/plot_comp0.pdf}
		\caption{$v$ exact vs. prediction}
		\label{fig:justb02b}
	\end{subfigure}
	\begin{subfigure}[b]{0.47\textwidth}
		\centering
		\includegraphics[scale=0.43]{../fitzhugh_nagumo_res_feature_onlyb_1/plot_comp1.pdf}
		\caption{$w$ exact vs. prediction}
		\label{fig:justb02c}
	\end{subfigure}
	\begin{subfigure}[b]{0.47\textwidth}
		\centering
		\includegraphics[scale=0.43]{../fitzhugh_nagumo_res_feature_onlyb_1/plot_pha.pdf}
		\caption{Phase space}
		\label{fig:justb02d}
	\end{subfigure}
	\caption{Prediction plots when $a=-0.3$, $b=1.1$, $\tau=20$ and $ I_{\text{ext}}=0.23$ after $10^5$ epochs. With feature transform $t \rightarrow \left[ t, \sin(0.013\cdot t) \right] $, and weights $\left[ \left[ 1, 1\right], \left[ 10, 10\right], \left[ 10^{-2}, 10^{-2}\right]\right]$.}
	%Label gjør det enkelt å referere til ulike bilder.
	\label{plot:justb02}
\end{figure} 	


As far as I can tell, figure \ref{plot:justb03} and \ref{plot:justb04} have exactly the hyperparameters, but one gives much better results. Maybe I copied the wrong file.

\begin{figure}[H]
	\centering 
	%Scale angir størrelsen på bildet. Bildefilen må ligge i %samme mappe som tex-filen. 
	\begin{subfigure}[b]{0.47\textwidth}
		\centering
		\includegraphics[scale=0.43]{../fitzhugh_nagumo_res_feature_onlyb_2/plot_pred.pdf}
		\caption{Prediction}
		\label{fig:justb03a}
	\end{subfigure}
	\begin{subfigure}[b]{0.47\textwidth}
		\centering
		\includegraphics[scale=0.43]{../fitzhugh_nagumo_res_feature_onlyb_2/plot_comp0.pdf}
		\caption{$v$ exact vs. prediction}
		\label{fig:justb03b}
	\end{subfigure}
	\begin{subfigure}[b]{0.47\textwidth}
		\centering
		\includegraphics[scale=0.43]{../fitzhugh_nagumo_res_feature_onlyb_2/plot_comp1.pdf}
		\caption{$w$ exact vs. prediction}
		\label{fig:justb03c}
	\end{subfigure}
	\begin{subfigure}[b]{0.47\textwidth}
		\centering
		\includegraphics[scale=0.43]{../fitzhugh_nagumo_res_feature_onlyb_2/plot_comp_nnb0.pdf}
		\caption{The NN's prediction in the best epoch}
		\label{fig:justb03d}
	\end{subfigure}
	\caption{Prediction plots when $a=-0.3$, $b=1.1$, $\tau=20$ and $ I_{\text{ext}}=0.23$ after $10^5$ epochs. With feature transform $t \rightarrow \left[ t, \sin(0.013\cdot t) \right] $, and weights $\left[ \left[ 1, 1\right], \left[ 10, 1\right], \left[ 0.1, 0.1\right]\right]$.}
	%Label gjør det enkelt å referere til ulike bilder.
	\label{plot:justb03}
\end{figure} 	



\begin{figure}[H]
	\centering 
	%Scale angir størrelsen på bildet. Bildefilen må ligge i %samme mappe som tex-filen. 
	\begin{subfigure}[b]{0.47\textwidth}
		\centering
		\includegraphics[scale=0.43]{../fitzhugh_nagumo_res_feature_onlyb_3/plot_pred.pdf}
		\caption{Prediction}
		\label{fig:justb04a}
	\end{subfigure}
	\begin{subfigure}[b]{0.47\textwidth}
		\centering
		\includegraphics[scale=0.43]{../fitzhugh_nagumo_res_feature_onlyb_3/plot_comp0.pdf}
		\caption{$v$ exact vs. prediction}
		\label{fig:justb04b}
	\end{subfigure}
	\begin{subfigure}[b]{0.47\textwidth}
		\centering
		\includegraphics[scale=0.43]{../fitzhugh_nagumo_res_feature_onlyb_3/plot_comp1.pdf}
		\caption{$w$ exact vs. prediction}
		\label{fig:justb04c}
	\end{subfigure}
	\begin{subfigure}[b]{0.47\textwidth}
		\centering
		\includegraphics[scale=0.43]{../fitzhugh_nagumo_res_feature_onlyb_3/plot_comp_nnb0.pdf}
		\caption{The NN's prediction in the best epoch}
		\label{fig:justb04d}
	\end{subfigure}
	\caption{Prediction plots when $a=-0.3$, $b=1.1$, $\tau=20$ and $ I_{\text{ext}}=0.23$ after $10^5$ epochs. With feature transform $t \rightarrow \left[ t, \sin(0.013\cdot t) \right] $, and weights $\left[ \left[ 1, 1\right], \left[ 1, 1\right], \left[ 0.1, 0.1 \right]\right]$.}
	%Label gjør det enkelt å referere til ulike bilder.
	\label{plot:justb04}
\end{figure} 	



\subsection{Weights}

In figure \ref{plot:justbode} we only have the ODE-weights. 

\begin{figure}[H]
	\centering 
	%Scale angir størrelsen på bildet. Bildefilen må ligge i %samme mappe som tex-filen. 
	\begin{subfigure}[b]{0.47\textwidth}
		\centering
		\includegraphics[scale=0.43]{../fitzhugh_nagumo_res_feature_onlyb_ode/plot_pred.pdf}
		\caption{Prediction}
		\label{fig:justbodea}
	\end{subfigure}
	\begin{subfigure}[b]{0.47\textwidth}
		\centering
		\includegraphics[scale=0.43]{../fitzhugh_nagumo_res_feature_onlyb_ode/plot_comp0.pdf}
		\caption{$v$ exact vs. prediction}
		\label{fig:justbodeb}
	\end{subfigure}
	\begin{subfigure}[b]{0.47\textwidth}
		\centering
		\includegraphics[scale=0.43]{../fitzhugh_nagumo_res_feature_onlyb_ode/plot_comp1.pdf}
		\caption{$w$ exact vs. prediction}
		\label{fig:justbodec}
	\end{subfigure}
	\begin{subfigure}[b]{0.47\textwidth}
		\centering
		\includegraphics[scale=0.43]{../fitzhugh_nagumo_res_feature_onlyb_ode/plot_comp_nnb0.pdf}
		\caption{The NN's prediction in the best epoch}
		\label{fig:justboded}
	\end{subfigure}
	\caption{Prediction plots when $a=-0.3$, $b=1.1$, $\tau=20$ and $ I_{\text{ext}}=0.23$ after $10^5$ epochs. With feature transform $t \rightarrow \left[ t, \sin(0.013\cdot t) \right] $, and weights $\left[ \left[ 0, 0\right], \left[ 0, 0\right], \left[ 1, 1\right]\right]$.}
	%Label gjør det enkelt å referere til ulike bilder.
	\label{plot:justbode}
\end{figure} 	


In figure \ref{plot:justbdata} we only have the data-weights. 

\begin{figure}[H]
	\centering 
	%Scale angir størrelsen på bildet. Bildefilen må ligge i %samme mappe som tex-filen. 
	\begin{subfigure}[b]{0.47\textwidth}
		\centering
		\includegraphics[scale=0.43]{../fitzhugh_nagumo_res_feature_onlyb_data/plot_pred.pdf}
		\caption{Prediction}
		\label{fig:justbdataa}
	\end{subfigure}
	\begin{subfigure}[b]{0.47\textwidth}
		\centering
		\includegraphics[scale=0.43]{../fitzhugh_nagumo_res_feature_onlyb_data/plot_comp0.pdf}
		\caption{$v$ exact vs. prediction}
		\label{fig:justbdatab}
	\end{subfigure}
	\begin{subfigure}[b]{0.47\textwidth}
		\centering
		\includegraphics[scale=0.43]{../fitzhugh_nagumo_res_feature_onlyb_data/plot_comp1.pdf}
		\caption{$w$ exact vs. prediction}
		\label{fig:justbdatac}
	\end{subfigure}
	\begin{subfigure}[b]{0.47\textwidth}
		\centering
		\includegraphics[scale=0.43]{../fitzhugh_nagumo_res_feature_onlyb_data/plot_comp_nnb0.pdf}
		\caption{The NN's prediction in the best epoch}
		\label{fig:justbdatad}
	\end{subfigure}
	\caption{Prediction plots when $a=-0.3$, $b=1.1$, $\tau=20$ and $ I_{\text{ext}}=0.23$ after $10^5$ epochs. With feature transform $t \rightarrow \left[ t, \sin(0.013\cdot t) \right] $, and weights $\left[ \left[ 0, 0\right],\left[ 1, 1\right],  \left[ 0, 0\right] \right]$.}
	%Label gjør det enkelt å referere til ulike bilder.
	\label{plot:justbdata}
\end{figure} 	


In figure \ref{plot:justbaux} we only have the BC-weights. 

\begin{figure}[H]
	\centering 
	%Scale angir størrelsen på bildet. Bildefilen må ligge i %samme mappe som tex-filen. 
	\begin{subfigure}[b]{0.47\textwidth}
		\centering
		\includegraphics[scale=0.43]{../fitzhugh_nagumo_res_feature_onlyb_bc/plot_pred.pdf}
		\caption{Prediction}
		\label{fig:justbauxa}
	\end{subfigure}
	\begin{subfigure}[b]{0.47\textwidth}
		\centering
		\includegraphics[scale=0.43]{../fitzhugh_nagumo_res_feature_onlyb_bc/plot_comp0.pdf}
		\caption{$v$ exact vs. prediction}
		\label{fig:justbauxb}
	\end{subfigure}
	\begin{subfigure}[b]{0.47\textwidth}
		\centering
		\includegraphics[scale=0.43]{../fitzhugh_nagumo_res_feature_onlyb_bc/plot_comp1.pdf}
		\caption{$w$ exact vs. prediction}
		\label{fig:justbauxc}
	\end{subfigure}
	\begin{subfigure}[b]{0.47\textwidth}
		\centering
		\includegraphics[scale=0.43]{../fitzhugh_nagumo_res_feature_onlyb_bc/plot_comp_nnb0.pdf}
		\caption{The NN's prediction in the best epoch}
		\label{fig:justbauxd}
	\end{subfigure}
	\caption{Prediction plots when $a=-0.3$, $b=1.1$, $\tau=20$ and $ I_{\text{ext}}=0.23$ after $10^5$ epochs. With feature transform $t \rightarrow \left[ t, \sin(0.013\cdot t) \right] $, and weights $\left[ \left[ 1, 1\right], \left[ 0, 0\right], \left[ 0, 0\right] \right]$.}
	%Label gjør det enkelt å referere til ulike bilder.
	\label{plot:justbaux}
\end{figure} 	



\subsection{More regular running}


A plot of the exact solution with $a=-0.3$, $b=1.1$, $\tau=20$ and $ I_{\text{ext}}=0.23$ can be seen in figure \ref{plot:exe0}.

\begin{figure}[H]
	\centering 
	%Scale angir størrelsen på bildet. Bildefilen må ligge i %samme mappe som tex-filen. 
	\includegraphics[scale=0.7]{../fitzhugh_nagumo_res_feature_onlyb_4/plot_exe.pdf}
	\caption{Plot of the exact solution when $a=-0.3$, $b=1.1$, $\tau=20$ and $ I_{\text{ext}}=0.23$}
	%Label gjør det enkelt å referere til ulike bilder.
	\label{plot:exe0}
\end{figure}

Figure \ref{plot:justb05} has the feature transform  $t \rightarrow \left[ \sin(0.013\cdot t) \right] $.

\begin{figure}[H]
	\centering 
	%Scale angir størrelsen på bildet. Bildefilen må ligge i %samme mappe som tex-filen. 
	\begin{subfigure}[b]{0.47\textwidth}
		\centering
		\includegraphics[scale=0.43]{../fitzhugh_nagumo_res_feature_onlyb_4/plot_pred.pdf}
		\caption{Prediction}
		\label{fig:justb05a}
	\end{subfigure}
	\begin{subfigure}[b]{0.47\textwidth}
		\centering
		\includegraphics[scale=0.43]{../fitzhugh_nagumo_res_feature_onlyb_4/plot_comp0.pdf}
		\caption{$v$ exact vs. prediction}
		\label{fig:justb05b}
	\end{subfigure}
	\begin{subfigure}[b]{0.47\textwidth}
		\centering
		\includegraphics[scale=0.43]{../fitzhugh_nagumo_res_feature_onlyb_4/plot_comp1.pdf}
		\caption{$w$ exact vs. prediction}
		\label{fig:justb05c}
	\end{subfigure}
	\begin{subfigure}[b]{0.47\textwidth}
		\centering
		\includegraphics[scale=0.43]{../fitzhugh_nagumo_res_feature_onlyb_4/plot_comp_nnb0.pdf}
		\caption{The NN's prediction in the best epoch}
		\label{fig:justb05d}
	\end{subfigure}
	\caption{Prediction plots when $a=-0.3$, $b=1.1$, $\tau=20$ and $ I_{\text{ext}}=0.23$ after $10^5$ epochs. With feature transform $t \rightarrow \left[ \sin(0.013\cdot t) \right] $, and weights $\left[ \left[ 1, 1\right], \left[ 10, 1\right], \left[ 0.1, 0.1 \right]\right]$.}
	%Label gjør det enkelt å referere til ulike bilder.
	\label{plot:justb05}
\end{figure} 	


\begin{figure}[H]
	\centering 
	%Scale angir størrelsen på bildet. Bildefilen må ligge i %samme mappe som tex-filen. 
	\begin{subfigure}[b]{0.47\textwidth}
		\centering
		\includegraphics[scale=0.43]{../fitzhugh_nagumo_res_feature_onlyb_5/plot_pred.pdf}
		\caption{Prediction}
		\label{fig:justb06a}
	\end{subfigure}
	\begin{subfigure}[b]{0.47\textwidth}
		\centering
		\includegraphics[scale=0.43]{../fitzhugh_nagumo_res_feature_onlyb_5/plot_comp0.pdf}
		\caption{$v$ exact vs. prediction}
		\label{fig:justb06b}
	\end{subfigure}
	\begin{subfigure}[b]{0.47\textwidth}
		\centering
		\includegraphics[scale=0.43]{../fitzhugh_nagumo_res_feature_onlyb_5/plot_comp1.pdf}
		\caption{$w$ exact vs. prediction}
		\label{fig:justb06c}
	\end{subfigure}
	\begin{subfigure}[b]{0.47\textwidth}
		\centering
		\includegraphics[scale=0.43]{../fitzhugh_nagumo_res_feature_onlyb_5/plot_comp_nnb0.pdf}
		\caption{The NN's prediction in the best epoch}
		\label{fig:justb06d}
	\end{subfigure}
	\caption{Prediction plots when $a=-0.3$, $b=1.1$, $\tau=20$ and $ I_{\text{ext}}=0.23$ after $10^5$ epochs. With feature transform $t \rightarrow \left[ t, \sin(0.013\cdot t) \right] $, and weights $\left[ \left[ 1, 1\right], \left[ 10, 1\right], \left[ 0.1, 0.1 \right]\right]$.}
	%Label gjør det enkelt å referere til ulike bilder.
	\label{plot:justb06}
\end{figure} 	


Here I realised 0.013 wouldn't work for $b=1.1$, so I used 0.0173 instead.

\begin{figure}[H]
	\centering 
	%Scale angir størrelsen på bildet. Bildefilen må ligge i %samme mappe som tex-filen. 
	\begin{subfigure}[b]{0.47\textwidth}
		\centering
		\includegraphics[scale=0.43]{../fitzhugh_nagumo_res_feature_onlyb_6/plot_pred.pdf}
		\caption{Prediction}
		\label{fig:justb07a}
	\end{subfigure}
	\begin{subfigure}[b]{0.47\textwidth}
		\centering
		\includegraphics[scale=0.43]{../fitzhugh_nagumo_res_feature_onlyb_6/plot_comp0.pdf}
		\caption{$v$ exact vs. prediction}
		\label{fig:justb07b}
	\end{subfigure}
	\begin{subfigure}[b]{0.47\textwidth}
		\centering
		\includegraphics[scale=0.43]{../fitzhugh_nagumo_res_feature_onlyb_6/plot_comp1.pdf}
		\caption{$w$ exact vs. prediction}
		\label{fig:justb07c}
	\end{subfigure}
	\begin{subfigure}[b]{0.47\textwidth}
		\centering
		\includegraphics[scale=0.43]{../fitzhugh_nagumo_res_feature_onlyb_6/plot_comp_nnb0.pdf}
		\caption{The NN's prediction in the best epoch}
		\label{fig:justb07d}
	\end{subfigure}
	\caption{Prediction plots when $a=-0.3$, $b=1.1$, $\tau=20$ and $ I_{\text{ext}}=0.23$ after $10^5$ epochs. With feature transform $t \rightarrow \left[ t, \sin(0.0173\cdot t) \right] $, and weights $\left[ \left[ 1, 1\right], \left[ 10, 1\right], \left[ 0.1, 0.1 \right]\right]$.}
	%Label gjør det enkelt å referere til ulike bilder.
	\label{plot:justb07}
\end{figure} 	


\subsection{No output scaling}

In figure \ref{plot:justb08} I tried removing the output scaling.

\begin{figure}[H]
	\centering 
	%Scale angir størrelsen på bildet. Bildefilen må ligge i %samme mappe som tex-filen. 
	\begin{subfigure}[b]{0.47\textwidth}
		\centering
		\includegraphics[scale=0.43]{../fitzhugh_nagumo_res_feature_onlyb_nooutput/plot_pred.pdf}
		\caption{Prediction}
		\label{fig:justb08a}
	\end{subfigure}
	\begin{subfigure}[b]{0.47\textwidth}
		\centering
		\includegraphics[scale=0.43]{../fitzhugh_nagumo_res_feature_onlyb_nooutput/plot_comp0.pdf}
		\caption{$v$ exact vs. prediction}
		\label{fig:justb08b}
	\end{subfigure}
	\begin{subfigure}[b]{0.47\textwidth}
		\centering
		\includegraphics[scale=0.43]{../fitzhugh_nagumo_res_feature_onlyb_nooutput/plot_comp1.pdf}
		\caption{$w$ exact vs. prediction}
		\label{fig:justb08c}
	\end{subfigure}
	\begin{subfigure}[b]{0.47\textwidth}
		\centering
		\includegraphics[scale=0.43]{../fitzhugh_nagumo_res_feature_onlyb_nooutput/plot_comp_nnb0.pdf}
		\caption{The NN's prediction in the best epoch}
		\label{fig:justb08d}
	\end{subfigure}
	\caption{Prediction plots when $a=-0.3$, $b=1.1$, $\tau=20$ and $ I_{\text{ext}}=0.23$ after $10^5$ epochs. With feature transform $t \rightarrow \left[ t, \sin(0.0173\cdot t) \right] $, and weights $\left[ \left[ 1, 1\right], \left[ 10, 1\right], \left[ 0.1, 0.1 \right]\right]$.}
	%Label gjør det enkelt å referere til ulike bilder.
	\label{plot:justb08}
\end{figure} 	


\begin{figure}[H]
	\centering 
	%Scale angir størrelsen på bildet. Bildefilen må ligge i %samme mappe som tex-filen. 
	\begin{subfigure}[b]{0.47\textwidth}
		\centering
		\includegraphics[scale=0.43]{../fitzhugh_nagumo_res_feature_onlyb_nooutput1/plot_pred.pdf}
		\caption{Prediction}
		\label{fig:justb09a}
	\end{subfigure}
	\begin{subfigure}[b]{0.47\textwidth}
		\centering
		\includegraphics[scale=0.43]{../fitzhugh_nagumo_res_feature_onlyb_nooutput1/plot_comp0.pdf}
		\caption{$v$ exact vs. prediction}
		\label{fig:justb09b}
	\end{subfigure}
	\begin{subfigure}[b]{0.47\textwidth}
		\centering
		\includegraphics[scale=0.43]{../fitzhugh_nagumo_res_feature_onlyb_nooutput1/plot_comp1.pdf}
		\caption{$w$ exact vs. prediction}
		\label{fig:justb09c}
	\end{subfigure}
	\begin{subfigure}[b]{0.47\textwidth}
		\centering
		\includegraphics[scale=0.43]{../fitzhugh_nagumo_res_feature_onlyb_nooutput1/plot_comp_nnb0.pdf}
		\caption{The NN's prediction in the best epoch}
		\label{fig:justb09d}
	\end{subfigure}
	\caption{Prediction plots when $a=-0.3$, $b=1.1$, $\tau=20$ and $ I_{\text{ext}}=0.23$ after $10^5$ epochs. With feature transform $t \rightarrow \left[ t, \sin(0.0173\cdot t) \right] $, and weights $\left[ \left[ 1, 1\right], \left[ 10, 1\right], \left[ 0.1, 0.1 \right]\right]$.}
	%Label gjør det enkelt å referere til ulike bilder.
	\label{plot:justb09}
\end{figure} 	



\end{document}

