\documentclass[a4paper]{article}
% Import some useful packages
\usepackage[margin=0.6in]{geometry} % narrow margins
\usepackage[utf8]{inputenc}
\usepackage[english]{babel}
\usepackage{hyperref}
\usepackage{listings}
\usepackage{amsmath,graphicx,varioref,verbatim,amsfonts,geometry,amssymb,dsfont,blindtext}
%\usepackage{minted}
\usepackage{amsmath}
\usepackage{xcolor}
\usepackage{booktabs}
\usepackage{epstopdf}
\usepackage{media9}
\usepackage{float}
\usepackage{caption}
\usepackage{subcaption}
\hypersetup{colorlinks=true}
\definecolor{LightGray}{gray}{0.95}

\definecolor{dkgreen}{rgb}{0,0.55,0}
\definecolor{blue}{rgb}{0,0,0.8}
\definecolor{gray}{rgb}{0.5,0.5,0.5}
\definecolor{mauve}{rgb}{0.58,0,0.82}
\definecolor{red}{rgb}{0.8,0,0}
\definecolor{mygray}{rgb}{0.96,0.96,0.96}
\definecolor{LightGray}{gray}{0.95}
\newcommand{\code}[1]{\colorbox{lightgray}{\texttt{#1}}}

\lstset{frame=tb,
	language=Python,
	aboveskip=3mm,
	belowskip=3mm,
	showstringspaces=false,
	columns=flexible,
	basicstyle={\small\ttfamily},
	numbers=none,
	otherkeywords={self,np,plt},
	numberstyle=\tiny\color{mauve},
	identifierstyle=\color{black},
	keywordstyle=\color{blue},
	commentstyle=\color{dkgreen},
	stringstyle=\color{red},
	backgroundcolor=\color{mygray},
	rulecolor=\color{black},
	breaklines=true,
	breakatwhitespace=true,
	%tabsize=3
	extendedchars=true,
	literate=
	{á}{{\'a}}1 {é}{{\'e}}1 {í}{{\'i}}1 {ó}{{\'o}}1 {ú}{{\'u}}1
	{Á}{{\'A}}1 {É}{{\'E}}1 {Í}{{\'I}}1 {Ó}{{\'O}}1 {Ú}{{\'U}}1
	{à}{{\`a}}1 {è}{{\`e}}1 {ì}{{\`i}}1 {ò}{{\`o}}1 {ù}{{\`u}}1
	{À}{{\`A}}1 {È}{{\'E}}1 {Ì}{{\`I}}1 {Ò}{{\`O}}1 {Ù}{{\`U}}1
	{ä}{{\"a}}1 {ë}{{\"e}}1 {ï}{{\"i}}1 {ö}{{\"o}}1 {ü}{{\"u}}1
	{Ä}{{\"A}}1 {Ë}{{\"E}}1 {Ï}{{\"I}}1 {Ö}{{\"O}}1 {Ü}{{\"U}}1
	{â}{{\^a}}1 {ê}{{\^e}}1 {î}{{\^i}}1 {ô}{{\^o}}1 {û}{{\^u}}1
	{Â}{{\^A}}1 {Ê}{{\^E}}1 {Î}{{\^I}}1 {Ô}{{\^O}}1 {Û}{{\^U}}1
	{œ}{{\oe}}1 {Œ}{{\OE}}1 {æ}{{\ae}}1 {Æ}{{\AE}}1 {ß}{{\ss}}1
	{ç}{{\c c}}1 {Ç}{{\c C}}1 {ø}{{\o}}1 {å}{{\r a}}1 {Å}{{\r A}}1
	{€}{{\EUR}}1 {£}{{\pounds}}1
}


%\title{Understanding the Code}
%\author{Bendik Steinsvåg Dalen}
\renewcommand\thesubsection{\thesection.\alph{subsection}}
\renewcommand\thesubsubsection{\thesubsection.\roman{subsubsection}}
\begin{document}
	
%[Work in progress].

All of the analysis here is done based on the python file which was copied into each respective folder. However you should keep in mind that I might have copied the wrong file at some point, especially for the earlier ones. 

\section{Fitting all parameters}

\subsection{Baseline}

At first we tried fitting all the parameters at the same time, with exact parameters $a=-0.3$, $b=1.4$, $\tau=20$ and $ I_{\text{ext}}=0.23$. A plot of the exact solution can be seen in figure \ref{plot:exe}.

\begin{figure}[H]
	\centering 
	%Scale angir størrelsen på bildet. Bildefilen må ligge i %samme mappe som tex-filen. 
	\includegraphics[scale=0.7]{../fitzhugh_nagumo_res_nonoise_1e5/plot_exe.pdf}
	\caption{Plot of the exact solution when $a=-0.3$, $b=1.4$, $\tau=20$ and $ I_{\text{ext}}=0.23$}
	%Label gjør det enkelt å referere til ulike bilder.
	\label{plot:exe}
\end{figure}


After $10^5$ epochs we got a prediction that flattened out instead of oscillating, see \ref{plot:all00}. 

\begin{figure}[H]
	\centering 
	%Scale angir størrelsen på bildet. Bildefilen må ligge i %samme mappe som tex-filen. 
	\begin{subfigure}[b]{0.47\textwidth}
		\centering
		\includegraphics[scale=0.46]{../fitzhugh_nagumo_res_nonoise_1e5/plot_pred.pdf}
		\caption{Prediction}
		\label{fig:all00a}
	\end{subfigure}
	\begin{subfigure}[b]{0.47\textwidth}
		\centering
		\includegraphics[scale=0.46]{../fitzhugh_nagumo_res_nonoise_1e5/plot_comp0.pdf}
		\caption{$v$ exact vs. prediction}
		\label{fig:all00b}
	\end{subfigure}
	\begin{subfigure}[b]{0.47\textwidth}
		\centering
		\includegraphics[scale=0.46]{../fitzhugh_nagumo_res_nonoise_1e5/plot_comp1.pdf}
		\caption{$w$ exact vs. prediction}
		\label{fig:all00c}
	\end{subfigure}
	\begin{subfigure}[b]{0.47\textwidth}
		\centering
		\includegraphics[scale=0.46]{../fitzhugh_nagumo_res_nonoise_1e5/plot_pha.pdf}
		\caption{Phase space}
		\label{fig:all00d}
	\end{subfigure}
	\caption{Prediction plots when $a=-0.3$, $b=1.4$, $\tau=20$ and $ I_{\text{ext}}=0.23$ after $10^5$ epochs. With feature transform $t, \sin(0.01 * t), \sin(0.05 * t), \sin(0.1 * t), \sin(0.15 * t)$.}
	%Label gjør det enkelt å referere til ulike bilder.
	\label{plot:all00}
\end{figure}


After $5\cdot10^5$ we got some overflow, which made the plots not appear, see \ref{plot:all01}. For these two first it seems like	I didn't save the original python file for some reason.	

\begin{figure}[H]
	\centering 
	%Scale angir størrelsen på bildet. Bildefilen må ligge i %samme mappe som tex-filen. 
	\begin{subfigure}[b]{0.47\textwidth}
		\centering
		\includegraphics[scale=0.46]{../fitzhugh_nagumo_res_nonoise_5e5/plot_pred.pdf}
		\caption{Prediction}
		\label{fig:all01a}
	\end{subfigure}
	\begin{subfigure}[b]{0.47\textwidth}
		\centering
		\includegraphics[scale=0.46]{../fitzhugh_nagumo_res_nonoise_5e5/plot_comp0.pdf}
		\caption{$v$ exact vs. prediction}
		\label{fig:all01b}
	\end{subfigure}
	\begin{subfigure}[b]{0.47\textwidth}
		\centering
		\includegraphics[scale=0.46]{../fitzhugh_nagumo_res_nonoise_5e5/plot_comp1.pdf}
		\caption{$w$ exact vs. prediction}
		\label{fig:all01c}
	\end{subfigure}
	\begin{subfigure}[b]{0.47\textwidth}
		\centering
		\includegraphics[scale=0.46]{../fitzhugh_nagumo_res_nonoise_5e5/plot_pha.pdf}
		\caption{Phase space}
		\label{fig:all01d}
	\end{subfigure}
	\caption{Prediction plots when $a=-0.3$, $b=1.4$, $\tau=20$ and $ I_{\text{ext}}=0.23$ after $5\cdot10^5$ epochs. With feature transform $t, \sin(0.01 * t), \sin(0.05 * t), \sin(0.1 * t), \sin(0.15 * t)$.}
	%Label gjør det enkelt å referere til ulike bilder.
	\label{plot:all01}
\end{figure} 	
	


In figure \ref{plot:all02} we got better results, but now with only $9\cdot10^4$ epochs. All the weights are set to 1. I'm not sure why we got better results with fewer epochs. It might just be the random state. 

\begin{figure}[H]
	\centering 
	%Scale angir størrelsen på bildet. Bildefilen må ligge i %samme mappe som tex-filen. 
	\begin{subfigure}[b]{0.47\textwidth}
		\centering
		\includegraphics[scale=0.46]{../fitzhugh_nagumo_res0/plot_pred.pdf}
		\caption{Prediction}
		\label{fig:all02a}
	\end{subfigure}
	\begin{subfigure}[b]{0.47\textwidth}
		\centering
		\includegraphics[scale=0.46]{../fitzhugh_nagumo_res0/plot_comp0.pdf}
		\caption{$v$ exact vs. prediction}
		\label{fig:all02b}
	\end{subfigure}
	\begin{subfigure}[b]{0.47\textwidth}
		\centering
		\includegraphics[scale=0.46]{../fitzhugh_nagumo_res0/plot_comp1.pdf}
		\caption{$w$ exact vs. prediction}
		\label{fig:all02c}
	\end{subfigure}
	\begin{subfigure}[b]{0.47\textwidth}
		\centering
		\includegraphics[scale=0.46]{../fitzhugh_nagumo_res0/plot_pha.pdf}
		\caption{Phase space}
		\label{fig:all02d}
	\end{subfigure}
	\caption{Prediction plots when $a=-0.3$, $b=1.4$, $\tau=20$ and $ I_{\text{ext}}=0.23$ after $9\cdot10^4$ epochs. With feature transform $t, \sin(0.01 * t), \sin(0.05 * t), \sin(0.1 * t), \sin(0.15 * t)$.}
	%Label gjør det enkelt å referere til ulike bilder.
	\label{plot:all02}
\end{figure}



\subsection{Different Scaling}

In figure \ref{plot:all03} we tried having the TensorFlow representation of the variables be scaled using the sigmoid function, instead of the soft plus function. In figure \ref{plot:all04} we tried a Gaussian distribution instead. The saved python file has $\tanh$ here, but I'm pretty sure it used Gaussian.

Other scaling methods were tried as well, including $\tanh$ and no scaling, but these lead to infinities, divide by zero or NaN. 

\begin{figure}[H]
	\centering 
	%Scale angir størrelsen på bildet. Bildefilen må ligge i %samme mappe som tex-filen. 
	\begin{subfigure}[b]{0.47\textwidth}
		\centering
		\includegraphics[scale=0.46]{../fitzhugh_nagumo_res_sigmoid/plot_pred.pdf}
		\caption{Prediction}
		\label{fig:all03a}
	\end{subfigure}
	\begin{subfigure}[b]{0.47\textwidth}
		\centering
		\includegraphics[scale=0.46]{../fitzhugh_nagumo_res_sigmoid/plot_comp0.pdf}
		\caption{$v$ exact vs. prediction}
		\label{fig:all03b}
	\end{subfigure}
	\begin{subfigure}[b]{0.47\textwidth}
		\centering
		\includegraphics[scale=0.46]{../fitzhugh_nagumo_res_sigmoid/plot_comp1.pdf}
		\caption{$w$ exact vs. prediction}
		\label{fig:all03c}
	\end{subfigure}
	\begin{subfigure}[b]{0.47\textwidth}
		\centering
		\includegraphics[scale=0.46]{../fitzhugh_nagumo_res_sigmoid/plot_pha.pdf}
		\caption{Phase space}
		\label{fig:all03d}
	\end{subfigure}
	\caption{Prediction plots when $a=-0.3$, $b=1.4$, $\tau=20$ and $ I_{\text{ext}}=0.23$ after $9\cdot10^4$ epochs. With feature transform $t, \sin(0.01 * t), \sin(0.05 * t), \sin(0.1 * t), \sin(0.15 * t)$.}
	%Label gjør det enkelt å referere til ulike bilder.
	\label{plot:all03}
\end{figure} 	


\begin{figure}[H]
	\centering 
	%Scale angir størrelsen på bildet. Bildefilen må ligge i %samme mappe som tex-filen. 
	\begin{subfigure}[b]{0.47\textwidth}
		\centering
		\includegraphics[scale=0.46]{../fitzhugh_nagumo_res_gaus/plot_pred.pdf}
		\caption{Prediction}
		\label{fig:all04a}
	\end{subfigure}
	\begin{subfigure}[b]{0.47\textwidth}
		\centering
		\includegraphics[scale=0.46]{../fitzhugh_nagumo_res_gaus/plot_comp0.pdf}
		\caption{$v$ exact vs. prediction}
		\label{fig:all04b}
	\end{subfigure}
	\begin{subfigure}[b]{0.47\textwidth}
		\centering
		\includegraphics[scale=0.46]{../fitzhugh_nagumo_res_gaus/plot_comp1.pdf}
		\caption{$w$ exact vs. prediction}
		\label{fig:all04c}
	\end{subfigure}
	\begin{subfigure}[b]{0.47\textwidth}
		\centering
		\includegraphics[scale=0.46]{../fitzhugh_nagumo_res_gaus/plot_pha.pdf}
		\caption{Phase space}
		\label{fig:all04d}
	\end{subfigure}
	\caption{Prediction plots when $a=-0.3$, $b=1.4$, $\tau=20$ and $ I_{\text{ext}}=0.23$ after $5\cdot10^4$ epochs. With feature transform $t, \sin(0.01 * t), \sin(0.05 * t), \sin(0.1 * t), \sin(0.15 * t)$.}
	%Label gjør det enkelt å referere til ulike bilder.
	\label{plot:all04}
\end{figure}



\subsubsection{Anchors}

In figure \ref{plot:all05} I tried to have the anchors only use part of \lstinline|data_t| by using the same indices as the randomly selected data. From what I remember it didn't have any noticeable difference, so I think I just used all of it from this point on. However, looking at the plot now it actually looks much worse than figure \ref{plot:all02}. That one was run for more epochs though, which might explain the difference. 

\begin{figure}[H]
	\centering 
	%Scale angir størrelsen på bildet. Bildefilen må ligge i %samme mappe som tex-filen. 
	\begin{subfigure}[b]{0.47\textwidth}
		\centering
		\includegraphics[scale=0.46]{../fitzhugh_nagumo_res_idx/plot_pred.pdf}
		\caption{Prediction}
		\label{fig:all05a}
	\end{subfigure}
	\begin{subfigure}[b]{0.47\textwidth}
		\centering
		\includegraphics[scale=0.46]{../fitzhugh_nagumo_res_idx/plot_comp0.pdf}
		\caption{$v$ exact vs. prediction}
		\label{fig:all05b}
	\end{subfigure}
	\begin{subfigure}[b]{0.47\textwidth}
		\centering
		\includegraphics[scale=0.46]{../fitzhugh_nagumo_res_idx/plot_comp1.pdf}
		\caption{$w$ exact vs. prediction}
		\label{fig:all05c}
	\end{subfigure}
	\begin{subfigure}[b]{0.47\textwidth}
		\centering
		\includegraphics[scale=0.46]{../fitzhugh_nagumo_res_idx/plot_pha.pdf}
		\caption{Phase space}
		\label{fig:all05d}
	\end{subfigure}
	\caption{Prediction plots when $a=-0.3$, $b=1.4$, $\tau=20$ and $ I_{\text{ext}}=0.23$ after $5\cdot10^4$ epochs. With feature transform $t, \sin(0.01 * t), \sin(0.05 * t), \sin(0.1 * t), \sin(0.15 * t)$.}
	%Label gjør det enkelt å referere til ulike bilder.
	\label{plot:all05}
\end{figure} 	



\subsubsection{No Features}

Figure \ref{plot:all06} shows the result with no feature transform. 

\begin{figure}[H]
	\centering 
	%Scale angir størrelsen på bildet. Bildefilen må ligge i %samme mappe som tex-filen. 
	\begin{subfigure}[b]{0.47\textwidth}
		\centering
		\includegraphics[scale=0.46]{../fitzhugh_nagumo_res_nofeature/plot_pred.pdf}
		\caption{Prediction}
		\label{fig:all06a}
	\end{subfigure}
	\begin{subfigure}[b]{0.47\textwidth}
		\centering
		\includegraphics[scale=0.46]{../fitzhugh_nagumo_res_nofeature/plot_comp0.pdf}
		\caption{$v$ exact vs. prediction}
		\label{fig:all06b}
	\end{subfigure}
	\begin{subfigure}[b]{0.47\textwidth}
		\centering
		\includegraphics[scale=0.46]{../fitzhugh_nagumo_res_nofeature/plot_comp1.pdf}
		\caption{$w$ exact vs. prediction}
		\label{fig:all06c}
	\end{subfigure}
	\begin{subfigure}[b]{0.47\textwidth}
		\centering
		\includegraphics[scale=0.46]{../fitzhugh_nagumo_res_nofeature/plot_pha.pdf}
		\caption{Phase space}
		\label{fig:all06d}
	\end{subfigure}
	\caption{Prediction plots when $a=-0.3$, $b=1.4$, $\tau=20$ and $ I_{\text{ext}}=0.23$ after $5\cdot10^4$ epochs. With feature transform $t, \sin(0.01 * t), \sin(0.05 * t), \sin(0.1 * t), \sin(0.15 * t)$.}
	%Label gjør det enkelt å referere til ulike bilder.
	\label{plot:all06}
\end{figure}



\subsubsection{Weights}

Figure \ref{plot:weight01} is the prediction with just the ODE-weights, the rest are set to zero.

\begin{figure}[H]
	\centering 
	%Scale angir størrelsen på bildet. Bildefilen må ligge i %samme mappe som tex-filen. 
	\begin{subfigure}[b]{0.47\textwidth}
		\centering
		\includegraphics[scale=0.46]{../fitzhugh_nagumo_res_justode/plot_pred.pdf}
		\caption{Prediction}
		\label{fig:weight01a}
	\end{subfigure}
	\begin{subfigure}[b]{0.47\textwidth}
		\centering
		\includegraphics[scale=0.46]{../fitzhugh_nagumo_res_justode/plot_comp0.pdf}
		\caption{$v$ exact vs. prediction}
		\label{fig:weight01b}
	\end{subfigure}
	\begin{subfigure}[b]{0.47\textwidth}
		\centering
		\includegraphics[scale=0.46]{../fitzhugh_nagumo_res_justode/plot_comp1.pdf}
		\caption{$w$ exact vs. prediction}
		\label{fig:weight01c}
	\end{subfigure}
	\begin{subfigure}[b]{0.47\textwidth}
		\centering
		\includegraphics[scale=0.46]{../fitzhugh_nagumo_res_justode/plot_pha.pdf}
		\caption{Phase space}
		\label{fig:weight01d}
	\end{subfigure}
	\caption{Prediction plots when $a=-0.3$, $b=1.4$, $\tau=20$ and $ I_{\text{ext}}=0.23$ after $5\cdot10^4$ epochs. With feature transform $t, \sin(0.01 * t), \sin(0.05 * t), \sin(0.1 * t), \sin(0.15 * t)$.}
	%Label gjør det enkelt å referere til ulike bilder.
	\label{plot:weight01}
\end{figure} 	

Figure \ref{plot:weight02} is the prediction with just the data-weights, the rest are set to zero.

\begin{figure}[H]
	\centering 
	%Scale angir størrelsen på bildet. Bildefilen må ligge i %samme mappe som tex-filen. 
	\begin{subfigure}[b]{0.47\textwidth}
		\centering
		\includegraphics[scale=0.46]{../fitzhugh_nagumo_res_justdata/plot_pred.pdf}
		\caption{Prediction}
		\label{fig:weight02a}
	\end{subfigure}
	\begin{subfigure}[b]{0.47\textwidth}
		\centering
		\includegraphics[scale=0.46]{../fitzhugh_nagumo_res_justdata/plot_comp0.pdf}
		\caption{$v$ exact vs. prediction}
		\label{fig:weight02b}
	\end{subfigure}
	\begin{subfigure}[b]{0.47\textwidth}
		\centering
		\includegraphics[scale=0.46]{../fitzhugh_nagumo_res_justdata/plot_comp1.pdf}
		\caption{$w$ exact vs. prediction}
		\label{fig:weight02c}
	\end{subfigure}
	\begin{subfigure}[b]{0.47\textwidth}
		\centering
		\includegraphics[scale=0.46]{../fitzhugh_nagumo_res_justdata/plot_pha.pdf}
		\caption{Phase space}
		\label{fig:weight02d}
	\end{subfigure}
	\caption{Prediction plots when $a=-0.3$, $b=1.4$, $\tau=20$ and $ I_{\text{ext}}=0.23$ after $5\cdot10^4$ epochs. With feature transform $t, \sin(0.01 * t), \sin(0.05 * t), \sin(0.1 * t), \sin(0.15 * t)$.}
	%Label gjør det enkelt å referere til ulike bilder.
	\label{plot:weight02}
\end{figure}

Figure \ref{plot:weight03} is the prediction with just the auxiliary-weights, the rest are set to zero.

\begin{figure}[H]
	\centering 
	%Scale angir størrelsen på bildet. Bildefilen må ligge i %samme mappe som tex-filen. 
	\begin{subfigure}[b]{0.47\textwidth}
		\centering
		\includegraphics[scale=0.46]{../fitzhugh_nagumo_res_justaux/plot_pred.pdf}
		\caption{Prediction}
		\label{fig:weight03a}
	\end{subfigure}
	\begin{subfigure}[b]{0.47\textwidth}
		\centering
		\includegraphics[scale=0.46]{../fitzhugh_nagumo_res_justaux/plot_comp0.pdf}
		\caption{$v$ exact vs. prediction}
		\label{fig:weight03b}
	\end{subfigure}
	\begin{subfigure}[b]{0.47\textwidth}
		\centering
		\includegraphics[scale=0.46]{../fitzhugh_nagumo_res_justaux/plot_comp1.pdf}
		\caption{$w$ exact vs. prediction}
		\label{fig:weight03c}
	\end{subfigure}
	\begin{subfigure}[b]{0.47\textwidth}
		\centering
		\includegraphics[scale=0.46]{../fitzhugh_nagumo_res_justaux/plot_pha.pdf}
		\caption{Phase space}
		\label{fig:weight03d}
	\end{subfigure}
	\caption{Prediction plots when $a=-0.3$, $b=1.4$, $\tau=20$ and $ I_{\text{ext}}=0.23$ after $9\cdot10^4$ epochs. With feature transform $t, \sin(0.01 * t), \sin(0.05 * t), \sin(0.1 * t), \sin(0.15 * t)$.}
	%Label gjør det enkelt å referere til ulike bilder.
	\label{plot:weight03}
\end{figure} 	

Figure \ref{plot:weight04} is the prediction with all weights set to zero.

\begin{figure}[H]
	\centering 
	%Scale angir størrelsen på bildet. Bildefilen må ligge i %samme mappe som tex-filen. 
	\begin{subfigure}[b]{0.47\textwidth}
		\centering
		\includegraphics[scale=0.46]{../fitzhugh_nagumo_res_zeroweights/plot_pred.pdf}
		\caption{Prediction}
		\label{fig:weight04a}
	\end{subfigure}
	\begin{subfigure}[b]{0.47\textwidth}
		\centering
		\includegraphics[scale=0.46]{../fitzhugh_nagumo_res_zeroweights/plot_comp0.pdf}
		\caption{$v$ exact vs. prediction}
		\label{fig:weight04b}
	\end{subfigure}
	\begin{subfigure}[b]{0.47\textwidth}
		\centering
		\includegraphics[scale=0.46]{../fitzhugh_nagumo_res_zeroweights/plot_comp1.pdf}
		\caption{$w$ exact vs. prediction}
		\label{fig:weight04c}
	\end{subfigure}
	\begin{subfigure}[b]{0.47\textwidth}
		\centering
		\includegraphics[scale=0.46]{../fitzhugh_nagumo_res_zeroweights/plot_pha.pdf}
		\caption{Phase space}
		\label{fig:weight04d}
	\end{subfigure}
	\caption{Prediction plots when $a=-0.3$, $b=1.4$, $\tau=20$ and $ I_{\text{ext}}=0.23$ after $5\cdot10^4$ epochs. With feature transform $t, \sin(0.01 * t), \sin(0.05 * t), \sin(0.1 * t), \sin(0.15 * t)$.}
	%Label gjør det enkelt å referere til ulike bilder.
	\label{plot:weight04}
\end{figure}


Figure \ref{plot:weight05} is the prediction with the ODE-weights set to zero, the rest are 1.

\begin{figure}[H]
	\centering 
	%Scale angir størrelsen på bildet. Bildefilen må ligge i %samme mappe som tex-filen. 
	\begin{subfigure}[b]{0.47\textwidth}
		\centering
		\includegraphics[scale=0.46]{../fitzhugh_nagumo_res_noode/plot_pred.pdf}
		\caption{Prediction}
		\label{fig:weight05a}
	\end{subfigure}
	\begin{subfigure}[b]{0.47\textwidth}
		\centering
		\includegraphics[scale=0.46]{../fitzhugh_nagumo_res_noode/plot_comp0.pdf}
		\caption{$v$ exact vs. prediction}
		\label{fig:weight05b}
	\end{subfigure}
	\begin{subfigure}[b]{0.47\textwidth}
		\centering
		\includegraphics[scale=0.46]{../fitzhugh_nagumo_res_noode/plot_comp1.pdf}
		\caption{$w$ exact vs. prediction}
		\label{fig:weight05c}
	\end{subfigure}
	\begin{subfigure}[b]{0.47\textwidth}
		\centering
		\includegraphics[scale=0.46]{../fitzhugh_nagumo_res_noode/plot_pha.pdf}
		\caption{Phase space}
		\label{fig:weight05d}
	\end{subfigure}
	\caption{Prediction plots when $a=-0.3$, $b=1.4$, $\tau=20$ and $ I_{\text{ext}}=0.23$ after $5\cdot10^4$ epochs. With feature transform $t, \sin(0.01 * t), \sin(0.05 * t), \sin(0.1 * t), \sin(0.15 * t)$.}
	%Label gjør det enkelt å referere til ulike bilder.
	\label{plot:weight05}
\end{figure} 	

Figure \ref{plot:weight06} is the prediction with the data-weights set to zero, the rest are 1.

\begin{figure}[H]
	\centering 
	%Scale angir størrelsen på bildet. Bildefilen må ligge i %samme mappe som tex-filen. 
	\begin{subfigure}[b]{0.47\textwidth}
		\centering
		\includegraphics[scale=0.46]{../fitzhugh_nagumo_res_nodata/plot_pred.pdf}
		\caption{Prediction}
		\label{fig:weight06a}
	\end{subfigure}
	\begin{subfigure}[b]{0.47\textwidth}
		\centering
		\includegraphics[scale=0.46]{../fitzhugh_nagumo_res_nodata/plot_comp0.pdf}
		\caption{$v$ exact vs. prediction}
		\label{fig:weight06b}
	\end{subfigure}
	\begin{subfigure}[b]{0.47\textwidth}
		\centering
		\includegraphics[scale=0.46]{../fitzhugh_nagumo_res_nodata/plot_comp1.pdf}
		\caption{$w$ exact vs. prediction}
		\label{fig:weight06c}
	\end{subfigure}
	\begin{subfigure}[b]{0.47\textwidth}
		\centering
		\includegraphics[scale=0.46]{../fitzhugh_nagumo_res_nodata/plot_pha.pdf}
		\caption{Phase space}
		\label{fig:weight06d}
	\end{subfigure}
	\caption{Prediction plots when $a=-0.3$, $b=1.4$, $\tau=20$ and $ I_{\text{ext}}=0.23$ after $5\cdot10^4$ epochs. With feature transform $t, \sin(0.01 * t), \sin(0.05 * t), \sin(0.1 * t), \sin(0.15 * t)$.}
	%Label gjør det enkelt å referere til ulike bilder.
	\label{plot:weight06}
\end{figure}

Figure \ref{plot:weight07} is the prediction with the auxiliary-weights set to zero, the rest are 1.

\begin{figure}[H]
	\centering 
	%Scale angir størrelsen på bildet. Bildefilen må ligge i %samme mappe som tex-filen. 
	\begin{subfigure}[b]{0.47\textwidth}
		\centering
		\includegraphics[scale=0.46]{../fitzhugh_nagumo_res_noaux/plot_pred.pdf}
		\caption{Prediction}
		\label{fig:weight07a}
	\end{subfigure}
	\begin{subfigure}[b]{0.47\textwidth}
		\centering
		\includegraphics[scale=0.46]{../fitzhugh_nagumo_res_noaux/plot_comp0.pdf}
		\caption{$v$ exact vs. prediction}
		\label{fig:weight07b}
	\end{subfigure}
	\begin{subfigure}[b]{0.47\textwidth}
		\centering
		\includegraphics[scale=0.46]{../fitzhugh_nagumo_res_noaux/plot_comp1.pdf}
		\caption{$w$ exact vs. prediction}
		\label{fig:weight07c}
	\end{subfigure}
	\begin{subfigure}[b]{0.47\textwidth}
		\centering
		\includegraphics[scale=0.46]{../fitzhugh_nagumo_res_noaux/plot_pha.pdf}
		\caption{Phase space}
		\label{fig:weight07d}
	\end{subfigure}
	\caption{Prediction plots when $a=-0.3$, $b=1.4$, $\tau=20$ and $ I_{\text{ext}}=0.23$ after $9\cdot10^4$ epochs. With feature transform $t, \sin(0.01 * t), \sin(0.05 * t), \sin(0.1 * t), \sin(0.15 * t)$.}
	%Label gjør det enkelt å referere til ulike bilder.
	\label{plot:weight07}
\end{figure} 	


\subsection{More Baseline}



\begin{figure}[H]
	\centering 
	%Scale angir størrelsen på bildet. Bildefilen må ligge i %samme mappe som tex-filen. 
	\begin{subfigure}[b]{0.47\textwidth}
		\centering
		\includegraphics[scale=0.46]{../fitzhugh_nagumo_res1/plot_pred.pdf}
		\caption{Prediction}
		\label{fig:all11a}
	\end{subfigure}
	\begin{subfigure}[b]{0.47\textwidth}
		\centering
		\includegraphics[scale=0.46]{../fitzhugh_nagumo_res1/plot_comp0.pdf}
		\caption{$v$ exact vs. prediction}
		\label{fig:all11b}
	\end{subfigure}
	\begin{subfigure}[b]{0.47\textwidth}
		\centering
		\includegraphics[scale=0.46]{../fitzhugh_nagumo_res1/plot_comp1.pdf}
		\caption{$w$ exact vs. prediction}
		\label{fig:all11c}
	\end{subfigure}
	\begin{subfigure}[b]{0.47\textwidth}
		\centering
		\includegraphics[scale=0.46]{../fitzhugh_nagumo_res1/plot_pha.pdf}
		\caption{Phase space}
		\label{fig:all11d}
	\end{subfigure}
	\caption{Prediction plots when $a=-0.3$, $b=1.4$, $\tau=20$ and $ I_{\text{ext}}=0.23$ after $9\cdot10^4$ epochs. With feature transform $t, \sin(0.01 * t), \sin(0.05 * t), \sin(0.1 * t), \sin(0.15 * t)$.}
	%Label gjør det enkelt å referere til ulike bilder.
	\label{plot:all11}
\end{figure}


\subsection{Negative $a$}

In figure \ref{plot:neg1} and \ref{plot:neg2} I tried forcing $a$ to be negative by setting the scaling multiplier to be negative.

\begin{figure}[H]
	\centering 
	%Scale angir størrelsen på bildet. Bildefilen må ligge i %samme mappe som tex-filen. 
	\begin{subfigure}[b]{0.47\textwidth}
		\centering
		\includegraphics[scale=0.46]{../fitzhugh_nagumo_neg/plot_pred.pdf}
		\caption{Prediction}
		\label{fig:neg1a}
	\end{subfigure}
	\begin{subfigure}[b]{0.47\textwidth}
		\centering
		\includegraphics[scale=0.46]{../fitzhugh_nagumo_neg/plot_comp0.pdf}
		\caption{$v$ exact vs. prediction}
		\label{fig:neg1b}
	\end{subfigure}
	\begin{subfigure}[b]{0.47\textwidth}
		\centering
		\includegraphics[scale=0.46]{../fitzhugh_nagumo_neg/plot_comp1.pdf}
		\caption{$w$ exact vs. prediction}
		\label{fig:neg1c}
	\end{subfigure}
	\begin{subfigure}[b]{0.47\textwidth}
		\centering
		\includegraphics[scale=0.46]{../fitzhugh_nagumo_neg/plot_pha.pdf}
		\caption{Phase space}
		\label{fig:neg1d}
	\end{subfigure}
	\caption{Prediction plots when $a=-0.3$, $b=1.4$, $\tau=20$ and $ I_{\text{ext}}=0.23$ after $9\cdot10^4$ epochs. With feature transform $t, \sin(0.01 * t), \sin(0.05 * t), \sin(0.1 * t), \sin(0.15 * t)$.}
	%Label gjør det enkelt å referere til ulike bilder.
	\label{plot:neg1}
\end{figure}

\begin{figure}[H]
	\centering 
	%Scale angir størrelsen på bildet. Bildefilen må ligge i %samme mappe som tex-filen. 
	\begin{subfigure}[b]{0.47\textwidth}
		\centering
		\includegraphics[scale=0.46]{../fitzhugh_nagumo_neg1/plot_pred.pdf}
		\caption{Prediction}
		\label{fig:neg2a}
	\end{subfigure}
	\begin{subfigure}[b]{0.47\textwidth}
		\centering
		\includegraphics[scale=0.46]{../fitzhugh_nagumo_neg1/plot_comp0.pdf}
		\caption{$v$ exact vs. prediction}
		\label{fig:neg2b}
	\end{subfigure}
	\begin{subfigure}[b]{0.47\textwidth}
		\centering
		\includegraphics[scale=0.46]{../fitzhugh_nagumo_neg1/plot_comp1.pdf}
		\caption{$w$ exact vs. prediction}
		\label{fig:neg2c}
	\end{subfigure}
	\begin{subfigure}[b]{0.47\textwidth}
		\centering
		\includegraphics[scale=0.46]{../fitzhugh_nagumo_neg1/plot_pha.pdf}
		\caption{Phase space}
		\label{fig:neg2d}
	\end{subfigure}
	\caption{Prediction plots when $a=-0.3$, $b=1.4$, $\tau=20$ and $ I_{\text{ext}}=0.23$ after $10^5$ epochs. With feature transform $t, \sin(0.01 * t), \sin(0.05 * t), \sin(0.1 * t), \sin(0.15 * t)$.}
	%Label gjør det enkelt å referere til ulike bilder.
	\label{plot:neg2}
\end{figure}


\subsection{Changing feature transform}





\section{Fitting $a$ and $b$}

In figure \ref{plot:ab0} we kept $\tau$ and $I_{\text{ext}}$ constant, and tried to only fit $a$ and $b$.
%something about the strange date? I think the result is the correct one, but the date when the results were gathered might be wrong. 

\begin{figure}[H]
	\centering 
	%Scale angir størrelsen på bildet. Bildefilen må ligge i %samme mappe som tex-filen. 
	\begin{subfigure}[b]{0.47\textwidth}
		\centering
		\includegraphics[scale=0.46]{../fitzhugh_nagumo_res_all_tIconst/plot_pred.pdf}
		\caption{Prediction}
		\label{fig:ab0a}
	\end{subfigure}
	\begin{subfigure}[b]{0.47\textwidth}
		\centering
		\includegraphics[scale=0.46]{../fitzhugh_nagumo_res_all_tIconst/plot_comp0.pdf}
		\caption{$v$ exact vs. prediction}
		\label{fig:ab0b}
	\end{subfigure}
	\begin{subfigure}[b]{0.47\textwidth}
		\centering
		\includegraphics[scale=0.46]{../fitzhugh_nagumo_res_all_tIconst/plot_comp1.pdf}
		\caption{$w$ exact vs. prediction}
		\label{fig:ab0c}
	\end{subfigure}
	\begin{subfigure}[b]{0.47\textwidth}
		\centering
		\includegraphics[scale=0.46]{../fitzhugh_nagumo_res_all_tIconst/plot_pha.pdf}
		\caption{Phase space}
		\label{fig:ab0d}
	\end{subfigure}
	\caption{Prediction plots when $a=-0.3$, $b=1.4$, $\tau=20$ and $ I_{\text{ext}}=0.23$ after $5\cdot10^4$ epochs. With feature transform $t, \sin(0.013*2*\\\pi*t)$.}
	%Label gjør det enkelt å referere til ulike bilder.
	\label{plot:ab0}
\end{figure}






\end{document}

